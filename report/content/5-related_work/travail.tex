\chapter{Travaux associés}
\label{travail associé}
Cette partie donne un aperçu de l'existant en matière d'apprentissage de la programmation.

Premièrement, il y aura un tour d'horizon des politiques gouvernementales de pays qui ont adopté l'apprentissage de la programmation dans leur programme officiel. Ceci présentera les avancées de cette problématique et aidera à envisager différentes manières de la traiter. 

Après ce tour d'horizon, l'attention sera portée sur des organisations non gouvernementales qui ont pour but de promouvoir l'enseignement de la programmation. 

La troisième partie décrira les langages utilisés dans les organisations présentées précédemment. 

Enfin, des concepts différenciateurs seront extraits de chaque initiative afin les distinguer l'une de l'autre.
\subsection{L'état de la programmation dans le monde}
\label{monde}
Cette partie fait un tour d'horizon de différent pays qui enseigne la programmation aux jeunes. Spécialement sur le positionnement politique que ces pays ont adopté par rapport à notre problématique. Cette partie ne porte donc plus sur une méthode ou une organisation, mais bien sûr le contexte et la manière d'intégré l'apprentissage de la programmation dans chaque pays.
\subsubsection{Angleterre}
L'apprentissage de l'informatique en Angleterre\footnote{\url{https://www.gov.uk/government/collections/statutory-guidance-schools\#national-curriculum-from-september-2014}} n'est pas nouveau. Cependant, longtemps cet apprentissage était centré sur les technologies de l'information et de la communication (TIC). En 2010, une étude a été commandée à la Royal Society pour évaluer cet apprentissage. Un an plus tard leur rapport a révélé que l'enseignement tel que dispensé jusque-là, n'était ni efficace ni en adéquation avec l'évolution de l'informatique dans notre société. La Royal Society suggère de changer les matières abordées en informatique. En effet précédemment c'était les TIC qui étaient enseignées alors que l'apprentissage de la programmation serait plus bénéfique et adapter pour les enfants. Sur base de ce rapport, les programmes de cours ont été adaptés.

\subsubsection{France}
En France\footnote{\url{http://fr.wikipedia.org/wiki/Informatique\_et\_sciences\_du\_num\%C3\%A9rique}
\url{http://fr.wikipedia.org/wiki/Baccalaur\%C3\%A9at\_scientifique}} depuis deux ans l'informatique fait partie intégrante du programme du baccalauréat de type S. Une des matières dispensées est "Informatique et sciences du numérique". Cette matière se subdivise en quatre sous parties qui sont : représentation de l'information, algorithmique, langage et programmation, architectures matérielles. Cette approche est donc également basée sur l'apprentissage de la programmation plutôt que sur les TIC.

\subsubsection{Nouvelle-Zélande}
Ce pays a adopté récemment les sciences informatiques dans son programme d'étude. Les cours sont dispensés à partir de 15 ans. Les cours dispensés concernent l'apprentissage de la programmation et des concepts informatique en général. Nous avons un choix un peu différent dans ce pays sur l'âge du début de l'apprentissage. Beaucoup de pays commencent plus jeunes et introduisent des concepts basiques. La Nouvelle-Zélande s'inscrit dans une logique plus similaire à la France, mais ne restreint pas l'informatique aux options scientifiques.

\subsubsection{Corée du Sud}
Enseigne l'informatique depuis longtemps et à tous les niveaux de l'enseignement. La culture numérique dans ces pays y est fort différente que par chez nous. Par exemple, une carrière dans le gaming y est tout à fait normale. L'informatique est vraiment omniprésente dans ces cultures, il est donc normal que son apprentissage commence en primaire.

\subsubsection{Grèce}
L'apprentissage des sciences informatiques prend une place importante dans les programmes grecs. Des 6 ans, les enfants sont confrontés à l'informatique à l'école. À cet âge, c'est plus de la maîtrise de l'outil qu'ils apprennent. Dès 10 ans, leurs cours d'informatique prennent une tournure plus algorithmique et donc plus proche de la science de l'informatique.

\section{Organisations existantes}
Il sera présenté ici un rassemblement d'initiatives similaires à ce travail. Ces initiatives sont toutes non-gouvernementales et propose d'enseigner la programmation de différente manières. C'est sur ce dernier point que l'accent sera mis dans la suite de ce tour des organisations.
\subsection{Code.org}
\begin{figure}[!h]
  \begin{center}
    \includegraphics[scale=0.5]{content/5-related_work/images/code}
    \caption{Logo de code.org}
    \label{fig:code.org}
  \end{center}
\end{figure}
Code.org\footnote{\url{http://code.org/about}} est une organisation sans but lucratif des USA qui à pour objectifs :
\begin{itemize}
  \item apporter l'informatique dans toutes les classes de secondaire aux USA ;
  \item démontrer le succès de l'utilisation de cours en ligne dans l'enseignement public ;
  \item ajouter l'informatique dans les bases des programmes de sciences et de mathématique des 50 états ;
  \item employé la connaissance technique collective pour améliorer l'apprentissage de l'informatique dans le monde ;
  \item augmenter la représentation féminine et des personnes de couleurs dans informatique.
\end{itemize}

Pour ce faire, ils fournissent une plate forme\footnote{\url{http://code.org/educate/20hr}} web qui permet aux professeurs de suivre leurs élèves grâce à un système de classe.

Toutes les ressources sont gratuites et librement utilisables\footnote{\url{http://code.org/faq}}. Leur programme d'apprentissage se base sur Blockly (voir \ref{blockly}).
Les ressources sont conçues pour que les professeurs comme les étudiants puissent commencer le cours sans connaître l'informatique (une assistance est proposée au professeur gratuitement si nécessaire).

Le site propose aux professeurs de se faire récompenser s'ils arrivent à finir les 27 missions proposées à minimum 15 étudiants. Dans ce cas, ils gagnent $750\$$, s'ils ont au moins 7 filles dans le groupe ils peuvent prétendre à $250\$$ supplémentaire.

\subsubsection{Déroulement des leçons}
Il propose des sessions de une heure de travail/jeu/apprentissage. Chaque unité d'une heure est découpée en petite mission (ex:5-20) les missions sont très courte et apporte un concept de programmation. Avant l'introduction de chaque concept, une petite vidéo est faite pour expliquer le concept introduit et des exemples de ce que la programmation permet de réaliser avec ce dernier.

Ils proposent de faire travailler les étudiants par pair\footnote{pair programming \url{http://en.wikipedia.org/wiki/Pair\_programming}}. Ceci permet d'avoir moins de questions pour le professeur et de mieux s'approprier la matière. Le travail par pair permet également de casser l'image du "geek" en montrant que la programmation est une science sociale et collaborative. Sans oublier qu'avec deux enfants par station, moins d'ordinateurs seront nécessaires.

Le site explique également que pour faire participer tout les élèves, il faut avoir confiance en leur compétence : permettre aux premiers groupes d'aider les derniers.

Quand un étudiant a une question, ils recommandent de proposer aux étudiants de d'abord demander à 3 de leurs camarades avant de poser la question au professeur. Le professeur ne devant pas être compétent, il doit juste pouvoir réfléchir avec les élèves de quel est le problème, cela permet aussi d'évité les questions de distraction ou de manque de compréhension.

Pour chaque petite mission, il y a un test automatisé qui dit si la mission est réussie ou non. Si la mission est réussie, le programme passe à la mission suivante. Il y a également un compteur de blocs dans les premières missions. Ce compteur permet de voir combien de blocs sont nécessaires pour réaliser la mission de manière optimale.

\subsection{CoderDojo}
\begin{figure}[!h]
  \begin{center}
    \includegraphics[scale=0.5]{content/5-related_work/images/dojo}
    \caption{Logo de CoderDojo}
    \label{fig:coder dojo}
  \end{center}
\end{figure}
CoderDojo\footnote{\url{http://coderdojo.com/about}} est un réseau open source de clubs de programmation dans le sens le plus large du terme. Tous les dojos sont donc autonomes. Dans ceux-ci, des enfants de 5 à 17 ans apprennent la programmation (site web, application, jeux...). La seul règle est "Above All : Be Cool" qui peut être mis en pratique simplement en créant des espaces d'échanges de savoir amical et sociable.

CoderDojo a été crée par James Whelton un irlandais de 18 ans et Bill Liao un entrepreneur australien à Cork. James a eu des demandes de jeunes enfants pour avoir des cours de programmation après qu'il eut hacké l'iPod nano. Beaucoup de gens de Dublin vinrent à ses cours et donc un nouveau Dojo a été créé à Dublin et puis cela s'est étendu à tout le globe.

\subsection{Code Club}
\begin{figure}[!h]
  \begin{center}
    \includegraphics[scale=0.3]{content/5-related_work/images/club}
    \caption{Logo de Code Club}
    \label{fig:code club}
  \end{center}
\end{figure}
Code Club\footnote{\url{https://www.codeclub.org.uk/about}} est un réseau de clubs national mené par des bénévoles en dehors des heures de cours. Ces activités s'adressent à des enfants de 9 à 11 ans.

Ils créent donc le matériel pour permettre à des bénévoles de donner des cours parascolaires d'environ une heure semaine. Il propose dans l'ordre d'utiliser scratch, HTML/css et puis Python. Ils aimeraient que les 21000 écoles primaires anglaises aient leur club.

Leur philosophie est de d'abord l'amusement, la créativité et l'exploration avant l'apprentissage des concepts de programmation.

\section{Langages de programmation}
\label{langages}
Cette partie traite des différents éditeurs et langages de programmation utilisés dans les organisations présentées au point précédent.
\subsection{Blockly}
\label{blockly}
\begin{figure}[!h]
  \begin{center}
    \includegraphics[scale=0.5]{content/5-related_work/images/blocky}
    \caption{Logo de Blocky}
    \label{fig:blocky}
  \end{center}
\end{figure}
Blockly\footnote{\url{https://code.google.com/p/blockly/}} est un langage de programmation graphique basé sur des technologies du web. 

Blockly est influencé\footnote{\url{https://code.google.com/p/blockly/wiki/Alternatives}} par "App Inventor" qui est influencé par "Scratch", lui-même influencé par "StarLogo"\footnote{\url{http://education.mit.edu/starlogo/}}.\\

Blockly a comme particularité :
\begin{itemize}
\item de s'exécuter dans un navigateur ;
\item d'exporter du code source en JavaScript, Dart, etc. ;
\item d'être open source ;
\item d'être haut niveau.
\end{itemize}

Il n'est pas directement une plateforme d'éducation dans le sens où, il peut être utilisé autant pour l'éducation, le business, des jeux, \ldots en fonction des blocs implémentés.\\

Blockly a été conçu avec certaines propriétés choisies lors de sa création. Les trois premières augmentent la compréhension des néophytes, les autres portent sur des facilités du langage. Les propriétés décidées lors de la conception du langage sont\footnote{\url{https://code.google.com/p/blockly/wiki/Language}} :

\begin{itemize}
  \item des indices de listes commençant à 1 ;
  \item des noms de variables non sensibles à la casse ;
  \item pas de portée de variable, elles sont toutes globales ;
  \item la possibilité de faire un export en JavaScript ;
  \item un code natif généré proche de celui des blocs.
\end{itemize}

\subsection{Scratch 1}
\begin{figure}[!h]
  \begin{center}
    \includegraphics[scale=0.4]{content/5-related_work/images/scratch}
    \caption{Logo de Scratch}
    \label{fig:scratch}
  \end{center}
\end{figure}
Scratch est un langage de programmation graphique développé par le MIT pour apprendre aux enfants la programmation. C'est l'interface qui permet de faire des scripts facilement grâce à une programmation par blocs et du glisser-déposer.\\

Il a été pensé pour être un outil créatif pour réaliser des histoires, des jeux, des simulations, de l'art, etc. Il a, par exemple, son propre éditeur d'image et de sons. Un autre but de ce langage est d'être simple à utiliser et à apprendre. Il a en effet été conçu pour des enfants n'ayant aucune connaissance préalable en programmation.\\

Actuellement Scratch est à sa version 2 qui est une version web. Cette version a été complètement réécrite en flash par rapport à la version 1 en Smalltalk. De plus, la version 2 n'est plus open source contrairement à la première version.

Comme le montre la figure \ref{fig:scratch-printscreen}, l'interface de Scratch se divise en plusieurs grandes parties :

\begin{enumerate}
\item sur la gauche, il y a la zone de dessin dans laquelle s'anime les composants graphiques des scriptes ;
\item au milieu, une liste des blocs disponible triée par catégorie ;
\item sur la droite, la zone de scripte qui contient tous les scriptes liés au sprite sélectionné.
\end{enumerate}
\begin{figure}[]
  \begin{center}
    \includegraphics[width=\textwidth]{content/5-related_work/images/scratch-printscreen}
    \caption{Interface de Scratch}
    \label{fig:scratch-printscreen}
  \end{center}
\end{figure}

\subsection{Snap!}
\begin{figure}[!h]
  \begin{center}
    \includegraphics[scale=0.07]{content/5-related_work/images/snap}
    \caption{Logo de Snap!}
    \label{fig:snap}
  \end{center}
\end{figure}
Snap! est un langage de programmation de "glissé-déposé" de blocs. C'est une ré-implémentation et une extension du langage Scratch du MIT. Il a été pensé et conçu pour être orienté web. Il est donc implémenté en JavaScript.\\

Ce langage est né en 2011 et a été créé par Jens Mönig, docteur de l'université de Berkeley. Il se distingue de son père Scratch par l'ajout :
\begin{enumerate}
\item de fonctions et de procédures de première classe ;
\item de listes de première classe ;
\item de sprite de première classe.
\end{enumerate}

\subsection{Python}
\begin{figure}[!h]
  \begin{center}
    \includegraphics[scale=0.4]{content/5-related_work/images/python}
    \caption{Logo de Python}
    \label{fig:python}
  \end{center}
\end{figure}
Python est un langage de programmation qu'il ne faut plus présenter. Il a beaucoup d'avantages dont celui d'avoir une syntaxe légère et d'être facile à prendre en main. Une grande communauté et beaucoup de bibliothèques, dont la fameuse \texttt{turtle}, en fait un excellent langage pour démarrer dans la programmation.

Cependant, devoir apprendre un langage de programmation et la logique informatique en même temps complique la tâche. De plus, lorsqu'on commence un nouveau langage de programmation il faut également apprendre ses bonnes pratiques de codage.

C'est donc un langage qui ne convient pas aux trop jeunes ni à ceux qui ne sont pas vraiment investis dans l'apprentissage de la programmation.

\section{Concepts différenciateurs}
\label{concepts}
Ce chapitre va mettre en lumière les différences et similitudes des méthodes d'apprentissage de la programmation en utilisant à une série de critères tels que le public cible, les moyens, le lieu et les personnes.
\subsection{Pour qui ?}
L'apprentissage de la programmation aux enfants est différenciée sur base de plusieurs critères tels que l'âge, l'origine, la classe sociale, le genre, etc. Les organisations auront donc soit un public visé restreint soit différents programmes pour les groupes d'enfants. 

\subsubsection{Âge}
La méthode d'apprentissages doit être différente suivant l'âge. Les enfants n'ont pas tous la maturité pour apprendre des concepts abstraits (boucle, parallélisation...). Par exemple, pour des élèves de maternel, un tableau interactif sera un grand plus car ils n'ont pas encore la dextérité souhaitée pour manipuler la souris. Et au contraire, des enfants plus grands seront plus autonomes et pourront travailler par binôme voir seuls.

\subsubsection{Genre et origine}
Certaines organisations, telle Code.org, mettent l'accent sur la accessibilité pour tous en incitant à la participation des filles qui sont sous-représentées en informatique. 

Les personnes qui ne peuvent acquérir un ordinateur ou se payer un stage ne doivent pas être délaissées.

\subsection{Comment ?}
Les principales différences entre toutes les organisations se situent dans la manière d'aborder l'apprentissage. Elles peuvent être de plusieurs ordres, tels que : les outils utilisés, les concepts abordés et l'environnement de travail.

\subsubsection{Outils} 
Un aspect pratique qui différencie les apprentissages est le type de langage utilisé. Ici encore différentes écoles s'affrontent.

\paragraph{Le langage réel} Avec ce type de langage, les enfants apprennent ce qu'est la programmation et donc ce qu'est du code source. De plus, avec un langage réel, plus de choses sont possibles.
\paragraph{Le langage visuel} Dans la programmation, le plus important est la logique. Donc, pourquoi ne pas avoir un langage visuel par blocs pour simplifier la syntaxe et se rapprocher un peu de ce qui se fait généralement avec des diagrammes?
\paragraph{Le langage web} Dans le monde d'aujourd'hui, le web est partout. L'accent doit donc être porté sur l'apprentissage de ce que les enfants voient au quotidien.


\subsubsection{Concepts} 
Tous les cours se divisent en deux parties : la théorie et la pratique. Il existe de multiples manières d'assembler les deux dans un cours unique. L'ordre dans lequel les concepts de programmation sont abordés peut se différencier sur plusieurs points.
\begin{itemize}
  \item Certains cours commencent par la théorie et demandent ensuite de l'appliquer ;
  \item D'autres proposent des exercices avant d'expliquer ce que les enfants ont utilisé de manière intuitive ;
  \item D'autres encore découpent la matière en petits sous-ensembles avant d'appliquer la première ou la seconde technique.\\
\end{itemize}


Ces différentes façons d'aborder l'apprentissage mènent diverses conceptions de missions. Celles-ci peuvent être très denses ou au contraire une succession de missions simples. Les missions denses sont souvent constituées de nombreux concepts de programmation (boucle, condition, fonction...) pour avoir directement un objectif concret alors que les missions simples essayent de bien séparer l'apprentissage pour fournir une étude la plus progressive possible.\\

Dans certains pays, la programmation est associée aux cours de sciences. Elle devient alors un outil scientifique qui perd du même coup une partie de son côté ludique. Par contre, elle touche un plus grand nombre d'enfants.\\

\subsubsection{Environement de travail}
\label{paire}
Les leçons peuvent être organisées de différentes manières :
\begin{description}
  \item[En groupe] L'apprentissage en groupe est souvent réalisé avec un tableau interactif ou un projecteur \footnote{\url{http://scratched.media.mit.edu/resources/scratch-\%C3\%A0-la-maternelle}} ;
  \item[Par binôme] La programmation en binôme est la plus répandue. Faire travailler les enfants par deux, les oblige à interagir entre eux et à essayer de résoudre eux-mêmes les problèmes qu'ils rencontrent. Cela permet aussi de diminuer le nombre d'ordinateurs utilisés ;
  \item[Individuel] Cette technique est très peu utilisé, car elle demande beaucoup de moyens matériels et personnels. De plus, lors d'un apprentissage individuel, les étudiants se retrouve vite coincé et demande de l'aide ;
  \item[A domicile] De nombreuses plateformes proposent des cours en ligne qui peuvent être suivi au rythme de chacun. Néanmoins, pour avoir de l'aide, il faut se tourner vers une personne compétente ou vers les forums. Ces formations sont souvent plus théoriques et demandent plus de lecture, ce qui empêche les jeunes enfants de les suivre.
\end{description}

\subsection{De qui ?}
Chaque initiative s'organise suivant différents critères tels que l'intégration ou non dans un programme scolaire, les compétences des encadrants et le système de gestion des ressources 

\subsubsection{Type d'organisation} 
Les cours d'informatique sont soit donnés dans le cadre d'un programme officiel soit en parascolaire.

Dans le premier cas, les étudiants ne viennent pas tous par choix. Ceci peut entraîner entre autres des blocages et des réticences à participer correctement à l'activité.

Pour les cours donnés en parascolaire, des petites structures très locales se partagent le terrain avec de grandes organisations à visées internationales. Le coût des activités et les horaires en sont les problèmes typiques.

\subsubsection{Encadrants} 
La plupart des cours sont donnés par des informaticiens ou au minimum des personnes formées à l'informatique. Mais comme le nombre d'informaticiens est réduit, certaines organisations proposent des cours clés-en-main pour des encadrants sans connaissance particulière en programmation. Ces professeurs apprennent en même temps que leurs étudiants et les aident à réfléchir.\\

Le type de rémunération des encadrants est assez diversifiée. Certains reçoivent des primes en fonction de la réussite de leur classe, d'autres sont payés par l'état ou directement par les étudiants, d'autres encore sont bénévoles.

\subsubsection{Gestion des ressources}
Suivant l'organisation, les cours sont plus ou moins figés. Certaines permettent à leur communauté de créer et/ou de modifier des cours tandis que d'autres ont des équipes dédiées à cette tâche.

