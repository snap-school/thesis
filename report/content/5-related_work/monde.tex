\section{État de la programmation dans le monde}
\label{monde}
Cette section fait un tour d'horizon de différents pays qui enseignent la programmation aux jeunes. L'accent est mis spécialement sur leur positionnement politique et leur manière d'intégrer l'apprentissage de la programmation. Ces pays ont été choisis sur base du rapport de l'académie des sciences de France \cite{ens-info-fr} et de la Royal Society \cite{comput-school}, car ils apportaient différentes conceptions et mise en application de l'apprentissage de la programmation.

\subsection{Angleterre}
L'apprentissage de l'informatique en Angleterre \cite{status-guid} n'est pas nouveau. Pendant longtemps, il était centré sur les technologies de l'information et de la communication (TIC). C'est donc principalement l'utilisation de l'informatique, et non sa conception qui était enseignée.

En 2010, une étude a été commandée à la Royal Society pour évaluer cet apprentissage. Un an plus tard, le rapport a révélé que l'enseignement n'est ni efficace ni en adéquation avec l'évolution de l'informatique. La Royal Society suggère d'adapter les matières abordées en informatique en enseignant l'apprentissage de la programmation. Sur base de ce rapport, les programmes de cours ont été modifiés en 2012.

\subsection{France}
En France \cite{wiki-info-et-sc-du-num} \cite{wiki-bac-sc}, depuis deux ans, l'informatique fait partie intégrante du programme du baccalauréat de type S. Une des matières dispensées est "Informatique et sciences du numérique". Cette matière se subdivise en quatre sous parties : représentation de l'information, algorithmique, langages de programmation et architectures matérielles. Cette approche est donc également basée sur l'apprentissage de la programmation plutôt que sur les TIC.

\subsection{Corée du Sud}
La Corée du Sud enseigne l'informatique depuis longtemps et à tous les niveaux de l'enseignement. La culture numérique dans les pays asiatiques est fort différente de ce que l'on connait chez nous. Par exemple, une carrière dans le jeu vidéo est quelque chose de tout à fait normal. L'informatique est vraiment omniprésente dans cette culture, il est donc normal que son apprentissage commence dès l'enseignement \gls{fondamental}.

\subsection{Grèce}
L'apprentissage des sciences informatiques prend une place importante dans les programmes grecs. Dès 6 ans, les enfants sont confrontés à l'informatique à l'école. À cet âge, ils apprennent principalement la maîtrise de l'outil. Dès 10 ans, leurs cours d'informatique prennent une tournure plus algorithmique et donc plus proche des sciences informatiques.

\subsection{Nouvelle-Zélande}
Ce pays a adopté récemment les sciences informatiques dans son programme d'étude. Les cours sont dispensés à partir de 15 ans. Ils comprennent l'apprentissage de la programmation et de concepts informatiques en général. La Nouvelle-Zélande, en introduisant les cours tardivement, s'inscrit dans une logique similaire à la France sans toutefois restreindre l'informatique aux options scientifiques.

\subsection{Belgique}
L'apprentissage de la programmation est déjà bien avancé dans plusieurs pays. Ce n'est malheureusement pas encore tout à fait le cas en Belgique. En effet, quelques initiatives locales existent, mais aucune décision politique n'a été prise jusqu'à présent.
