\section{Concepts différenciateurs}
\label{concepts}
Ce chapitre va mettre en lumière les différences et similitudes des méthodes d'apprentissage de la programmation en utilisant une série de critères tels que le public cible, les moyens, le lieu et les personnes.
\subsection{Pour qui ?}
L'apprentissage de la programmation aux enfants est différencié sur base de plusieurs critères tels que l'âge, l'origine, la classe sociale, le genre, etc. Les organisations auront donc soit un public visé restreint soit différents programmes pour les groupes d'enfants.

\subsubsection{Âge}
La méthode d'apprentissage doit être différente suivant l'âge. Les enfants n'ont pas tous la maturité pour apprendre des concepts abstraits (boucle, parallélisation...). Par exemple, pour des élèves de maternelle, un tableau interactif sera un grand plus, car ils n'ont pas encore la dextérité souhaitée pour manipuler la souris. Et au contraire, des enfants plus grands seront plus autonomes et pourront travailler par binôme voir seul.

\subsubsection{Genre et origine}
Certaines organisations, telle Code.org, mettent l'accent sur l’accessibilité pour tous en incitant à la participation des filles qui sont sous-représentées en informatique.

Les personnes qui ne peuvent acquérir un ordinateur ou se payer un stage ne doivent pas être délaissées.

\subsection{Comment ?}
Les principales différences entre toutes les organisations se situent dans la manière d'aborder l'apprentissage. Elles peuvent être de plusieurs ordres, tels que : les outils utilisés, les concepts abordés et l'environnement de travail.

\subsubsection{Outils}
Un aspect pratique qui différencie les apprentissages est le type de langage utilisé. Ici encore différentes écoles s'affrontent.

\paragraph{Le langage réel} Avec ce type de langage, les enfants apprennent la programmation et le langage utilisé. Ces langages sont plus difficiles à prendre en mains, mais permettent plus d'applications concrètes.
\paragraph{Le langage visuel} Dans la programmation, le plus important est la logique. Un langage qui se rapproche des diagrammes allège la syntaxe, par exemple, avec un langage de programmation par \glspl{bloc}.
\paragraph{Le langage web} Dans le monde d'aujourd'hui, le web est partout. L'accent doit donc être porté sur l'apprentissage de ce que les enfants voient au quotidien.


\subsubsection{Concepts}
Tous les cours se divisent en deux parties : la théorie et la pratique. Il existe de multiples manières d'assembler les deux dans un cours unique. L'ordre dans lequel les concepts de programmation sont abordés peut se différencier sur plusieurs points:
\begin{itemize}
  \item certains cours commencent par la théorie et demandent ensuite de l'appliquer ;
  \item d'autres proposent des exercices avant d'expliquer ce que les enfants ont utilisé de manière intuitive ;
  \item d'autres encore découpent la matière en petits sous-ensembles avant d'appliquer la première ou la seconde technique.\\
\end{itemize}


Ces différentes façons d'aborder l'apprentissage mènent à diverses conceptions de \glspl{mission}. Celles-ci peuvent être très denses ou au contraire une succession de \glspl{mission} simples. Les \glspl{mission} denses sont souvent constituées de nombreux concepts de programmation (boucle, condition, fonction...) pour avoir directement un objectif concret alors que les \glspl{mission} simples essayent de bien séparer l'apprentissage pour fournir une étude la plus progressive possible.\\

Dans certains pays, la programmation est associée aux cours de sciences. Elle devient alors un outil qui perd du même coup une partie de son côté ludique. Par contre, elle touche un plus grand nombre d'enfants.\\

\subsubsection{Environement de travail}
\label{paire}
Les leçons peuvent être organisées de différentes manières :
\begin{description}
  \item[En groupe] L'apprentissage en groupe est souvent réalisé avec un tableau interactif ou un projecteur \footnote{\url{http://scratched.media.mit.edu/resources/scratch-\%C3\%A0-la-maternelle}} ;
  \item[Par binôme] La programmation en binôme est la plus répandue. Faire travailler les enfants par deux les obligent à interagir entre eux et à essayer de résoudre eux-mêmes les problèmes qu'ils rencontrent. Cela permet aussi de diminuer le nombre d'ordinateurs utilisés ;
  \item[Individuel] Cette technique est très peu utilisée, car elle demande beaucoup de moyens matériels et personnels. De plus, lors d'un apprentissage individuel, les étudiants se retrouvent vite coincés et demandent de l'aide ;
  \item[A domicile] De nombreuses plateformes proposent des cours en ligne qui peuvent être suivis au rythme de chacun. Néanmoins, pour avoir de l'aide, il faut se tourner vers une personne compétente ou vers les forums. Ces formations sont souvent plus théoriques et demandent plus de lecture, ce qui empêche les jeunes enfants de les suivre.
\end{description}

\subsection{De qui ?}
Chaque initiative s'organise suivant différents critères tels que l'intégration ou non dans un programme scolaire, les compétences des encadrants et le système de gestion des ressources

\subsubsection{Type d'organisation}
Les cours d'informatique sont soit donnés dans le cadre d'un programme officiel soit en parascolaire.

Dans le premier cas, les étudiants ne viennent pas tous par choix. Ceci peut entraîner entre autres des blocages et des réticences à participer correctement à l'activité.

Pour les cours donnés en parascolaire, des petites structures très locales se partagent le terrain avec de grandes organisations à visées internationales. Le coût des activités et les horaires en sont les problèmes typiques.

\subsubsection{Encadrants}
La plupart des cours sont donnés par des informaticiens ou au minimum des personnes formées à l'informatique. Mais comme le nombre d'informaticiens est réduit, certaines organisations proposent des cours clés en main pour des encadrants sans connaissance particulière en programmation. Ces professeurs apprennent en même temps que leurs étudiants et les aident à réfléchir.\\

Le type de rémunération des encadrants est assez diversifié. Certains reçoivent des primes en fonction de la réussite de leur classe, d'autres sont payés par l'état ou directement par les étudiants, d'autres encore sont bénévoles.

\subsubsection{Gestion des ressources}
Suivant l'organisation, les cours sont plus ou moins figés. Certaines permettent à leur communauté de créer et/ou de modifier des cours tandis que d'autres ont des équipes dédiées à cette tâche.
