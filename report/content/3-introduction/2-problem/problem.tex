\section{Problème}
\label{into-problem}
Le but de ce travail est de promouvoir l'apprentissage de la programmation auprès des plus jeunes. En effet, comme annoncé dans la section \ref{intro-context} qu'il y a un manque de moyens pédagogiques dans ce domaine. 

Ce travail propose une adaptation d'un programme informatique existant Snap! mais l'intégrant dans une interface web. 

La volonté est de fournir une plateforme......
Ce travail vise donc à produire une nouvelle plateforme spécialisée dans l'apprentissage de la programmation accessible à tous, aussi bien pour les professeurs que les élèves. L'enseignant pourra grâce à l'application donner des cours de programmation à ses classes sans avoir de connaissances particulières.

Fournir.....
En plus de l'application, un ensemble de missions simple a été réalisé et testé sur des enfants pour montrer sa viabilité. Les missions proposées devaient avoir aussi bien des objectifs pédagogiques que ludiques.
