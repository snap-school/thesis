\section{Problème}
\label{into-problem}
Le but de ce travail est de promouvoir l'apprentissage de la programmation auprès des plus jeunes. En effet là nous avons vu dans la section \ref{intro-context} qu'il y a un manque de moyens pédagogiques dans ce domaine. Ce travail propose une adaptation d'un programme informatique existant Snap! mais en amenant cela dans une interface web et soutenu par un serveur. Ce travail vise donc à produire une nouvelle plateforme spécialisée dans l'apprentissage de la programmation accessible à tous, aussi bien les professeurs que les élèves. L'enseignant pourra grâce à l'application donner des cours de programmation à ses classes sans avoir de connaissances particulières.

En plus de l'application, un ensemble de missions simple a été réalisé et testé sur des enfants pour montrer que ce projet est viable. Les missions proposées devaient avoir aussi bien des objectifs pédagogiques que ludiques.
