\section{Plan du mémoire}
La suite de ce mémoire se divise de la manière suivante:

\paragraph{Connaissances nécessaire}
Ce chapitre introduit le bagage technique nécessaire à la bonne compréhension de ce travail. Il commence par quelques prérequis, puis introduit Snap! et enfin parle plus en détail de Ruby on Rails.

\paragraph{Travaux associés}
Ce chapitre parle de l'état de l'art dans le monde. Il présente l'avancée de l'apprentissage de la programmation dans le monde. Ensuite, il présente quelques grandes initiatives similaires à celle de ce travail. Après quoi, il introduit les différents langages de programmations enseignés aux jeunes. Et se finit par une liste de concepts différenciateurs qui permet de définir une initiative de manière unique.

\paragraph{Définition de la problématique}
Ce chapitre présente la position de ce mémoire par rapport au chapitre précédent. Il positionne Rsnap par rapport au concepts différenciateurs introduits dans les travaux associés. Et pour finir, il décrit les choix technologiques faits dans le cadre de ce travail.

\paragraph{Solution}
Ce chapitre décrit la solution que ce mémoire apporte. Il commence par décrire les missions qui ont été crées pour ce travail. Il parle ensuite des modifications apportées à Snap! pour les besoins du mémoire. Il fini par décrire la partie web de Rsnap.

\paragraph{Validation}
Ce chapitre présente la validation de l'application. Il commence par les tests sur l'application Rsnap. Il présente ensuite les expériences qui ont été menées. Et fini par les formulaires, leurs créations et leurs analyses.

\paragraph{Conclusion}
Ce chapitre conclu ce travail, compare les objectifs attendus et atteints, les points plus faibles et les points fort de ce mémoire.

\paragraph{Futures amélioration}
Ce chapitre décrit ce qui serait utile de faire dans la continuité de ce travail. Ce sont des améliorations qui n'ont pas pues être réalisées soit par manque de temps soit parce qu'elles sont apparues dans l'analyse des expériences.

%Dans cette section seront présentés les prérequis et un état des lieux de l'apprentissage de la programmation aux enfants dans le monde. En premier lieu, il y aura les prérequis et SNAP BYOB. Ensuite un analyse d'autres initiatives similaires à celle de ce travail. Suite à cela vendra la position du projet RSNAP par rapport à ce qui est déjà existant et expliqué dans les points précédents, suivis d'une présentation des outils utilisés. La dernière partie portera sur ce qu'apporte RSNAP par rapport aux autres projets existants. 
