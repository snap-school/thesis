\section{Plan du mémoire}
La suite de ce mémoire se divise de la manière suivante:

\paragraph{Connaissances nécessaire}
Ce chapitre introduit le bagage technique nécessaire à la bonne compréhension de ce travail. Il commence par quelques prérequis, puis introduit \gls{snap} et enfin parle plus en détail de \gls{rails}.

\paragraph{Travaux associés}
Ce chapitre parle de l'état de l'art dans le monde. Il présente l'avancée de l'apprentissage de la programmation dans le monde. Ensuite, il présente quelques grandes initiatives similaires à celle de ce travail. Après quoi, il introduit les différents langages de programmations enseignés aux jeunes. Et se finit par une liste de concepts différenciateurs qui permet de définir une initiative de manière unique.

\paragraph{Définition de la problématique}
Ce chapitre présente la position de ce mémoire par rapport au chapitre précédent. Il positionne \gls{rsnap} par rapport au concepts différenciateurs introduits dans les travaux associés. Et pour finir, il décrit les choix technologiques faits dans le cadre de ce travail.

\paragraph{Solution}
Ce chapitre décrit la plateforme \gls{rsnap} implémentée dans le cadre de ce mémoire.
Il commence par décrire les \glspl{mission} qui ont été crées pour ce travail. Il parle ensuite des modifications apportées à \gls{snap} pour les besoins du mémoire. Il fini par décrire la partie web de \gls{rsnap}.

\paragraph{Validation}
Ce chapitre présente la validation de l'application. Il commence par les tests sur l'application \gls{rsnap}. Il présente ensuite les expériences qui ont été menées. Et fini par l'analyse des formulaires proposé aux participants de l'expérience.

\paragraph{Conclusion}
Ce chapitre conclu ce travail, compare les objectifs attendus et atteints, les points plus faibles et les points fort de ce mémoire. %TODO à retravailler quand elle sera écrite

\paragraph{Futures amélioration}
Ce chapitre décrit ce qui serait utile de faire dans la continuité de ce travail. Ce sont des améliorations qui n'ont pas pues être réalisées soit par manque de temps soit parce qu'elles sont apparues dans l'analyse des expériences.
