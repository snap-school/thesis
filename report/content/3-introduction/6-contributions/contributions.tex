\section{Contributions}
\label{intro-contribution}
Cette section présente les différentes contributions de ce travail. Les trois principales sont : un outil francophone pour apprendre la programmation, une plateforme intégrée orientée pour l'enseignement et la combinaison d'outils existants.

\paragraph{Un outil francophone :} Comme introduit dans la section \ref{into-problem} il n'existait pas encore ou peu de matériel francophone pour soutenir l'apprentissage de la programmation. \gls{rsnap} est une plateforme entièrement francophone.%TODO pour moi on peut supprimer la suite : et les traductions de \gls{snap} ont été finie et améliorée dans le cadre de ce travail.

\paragraph{Une plateforme d'apprentissage :} Aucune plateforme n'est actuellement disponible pour l'apprentissage de la programmation. \gls{rsnap} se veut être orienté pour l'apprentissage dans l'enseignement traditionnel. Ceci en intégrant une gestion de classe, une progression contrôlable par l'enseignant. \gls{rsnap} a été pensée dès le début pour que les élèves ainsi que les professeurs puissent l'utiliser de la manière la plus autonome possible.

\paragraph{Combinaison d'outils existants :} Ce travail montre que de nombreuses applications sont déjà disponibles. Malheureusement, elles sont souvent restreintes à leur domaine d'application tel que les langages informatiques ou les plateformes web. La force de ce mémoire a été de combiner ces applications afin qu'elles se complètent pour devenir une application plus vaste.
