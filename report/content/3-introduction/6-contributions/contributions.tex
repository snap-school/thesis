\section{Contributions}
\label{intro-contribution}
Cette section présente les différentes contribution de ce travail à l'existant. Nous retiendront principalement: un outils francophone pour apprendre la programmation, une plateforme intégrée orienté pour l'enseignement et la combinaison d'outils existant.

\paragraph{Un outil francophone} Comme introduit dans la section \ref{into-problem} il n'existait pas encore ou peu de matériel francophone pour soutenir l'apprentissage de la programmation. \gls{rsnap} est une plateforme entièrement francophone.%TODO pour moi on peut supprimer la suite : et les traductions de \gls{snap} on été finie et améliorée dans le cadre de ce travail.

\paragraph{Une plateforme d'apprentissage} Aucune plateforme n'est actuellement disponible pour l'apprentissage de la programmation. \gls{rsnap} se veut être orienté pour l'apprentissage dans l'enseignement traditionnel. Ceci en intégrant une gestion de classe, une progression contrôlable par l'enseignant. \gls{rsnap} à été pensée dès le début pour que les élèves ainsi que les proffesseurs puissent l'utiliser de la manière la plus autonome possible.

\paragraph{Combinaison d'outils existant} Ce travaille montre que de nombreuses applications sont déjà disponible. Malheureusement, elles sont souvent restreinte à leur domaine d'application tel que les langages informatique ou les plateformes web. La force de ce mémoire a été de combiner ces applications afin qu'elles se complète pour devenir une application plus vaste.
