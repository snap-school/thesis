\section{Contributions}
\label{intro-contribution}
Cette section présente les différentes contribution de ce travail à l'existant. Nous retiendront principalement: un outils francophone pour apprendre la programmation, une plateforme intégrée orienté pour l'enseignement et la combinaison d'outils existant.

\paragraph{Un outil francophone} Comme introduit dans la section \ref{into-problem} il n'existait pas encore ou peu de matériel francophone pour soutenir l'apprentissage de la programmation. \gls{Rsnap} est une plateforme entièrement francophone.% et les traductions de Snap! on été finie et améliorée dans le cadre de ce travail.

\paragraph{Une plateforme d'apprentissage} Aucune plateforme n'est actuellement disponible pour l'apprentissage de la programmation. \gls{Rsnap} se veut être orienté pour l'apprentissage dans l'enseignement traditionnel. Ceci en intégrant une gestion de classe, une progression contrôlable par l'enseignant. \gls{Rsnap} à été pensée dès le début pour que les élèves ainsi que les proffesseurs puissent l'utiliser de la manière la plus autonome possible.

\paragraph{Combinaison d'outils existant} Ce travaille montre que de nombreuses applications sont déjà disponible. Malheureusement, elles sont souvent restreinte à leur domaine d'application tel que les langages informatique ou les plateformes web. La force de ce mémoire a été de combiner ces applications afin qu'elles se complète pour devenir une application plus vaste.


%
%
%
%
%%TODO Programmation francophone, plateforme, montrer que c'est facile qu'il y a deja plein d'outils,
%Papier sur les biens faits de l'apprentissage de la programmation sur l'esprit logique.
%
%Le projet SNAP est à l'origine maintenu par jmoening. C'est la principale contribution au projet. Dans ce travail, certaines parties ont pu être intégrées au projet, par exemple la traduction des aides en français a pu être ajoutée au projet de jmoening. Deux autres pull request sont en court. La première concerne l'ajout de rôle pour pouvoir modifier l'interface facilement en fonction des rôles que l'on souhaite avoir. La seconde porte sur une traduction automatisée de l'interface. Une autre pourrait encore être proposée pour les fonctions d'importation et d'exportation qui ont été ajoutées pour ce travail.
%
%Jimoegning bref lui :)
%
%Tous les autres intervenants
%
%Écoles ?
%
%Printemps des sciences ?
