\section{Objectifs}
\label{intro-objectifs}

Deux objectif principaux sont poursuivit par ce travail:

\paragraph{Une plateforme d'apprentissage} Fournir une plateforme qui permette d'enseigner la programmation aux jeunes. Cette plateforme doit être la plus facile d'accès et d'utilisation possible. Cette plateforme doit pouvoir stocker des informations pour qu'elles soient disponible lors d'une prochaine utilisation.

\paragraph{Du contenu} Fournir du contenu à la plateforme définie plus haut. A savoir des exercices qui ont pour but d'introduire la programmation chez les jeunes. Ces exercice doivent être validés sur des jeunes volontaire afin d'avoir la meilleur qualité de contenu. La qualité du contenu initial est très important car il servira d'exemple pour la création future de contenu.

Au delàs de ces deux objectifs, un troisième a déjà été annoncé plus tôt dans ce rapport. Ce travail est a destination d'un publique francophone, il est donc primordial que la plateforme comme le contenu soit en français.

%TODO abstraire
Le but de ce travail est de fournir une interface pour apprendre aux jeunes la programmation. Cette interface devra être dans un navigateur pour des questions de facilité d'utilisation. En effet, cette application est orientée vers les écoles et donc il faut une utilisation simple et intuitive faite pour des personnes n'ayant pas ou peu de connaissances en programmation et le minimum en informatique en général. L'intérêt d'une plateforme web est aussi de ne devoir rien installer sur les ordinateurs.

Il faut également deux types d'interface, une pour les jeunes où toutes les fonctions inutiles sont retirées ainsi que les options pour modifier l'interface. Une pour les professeurs pour leur permettre d'avoir accès aux fonctions qui permettent de cacher des blocs et des scripts.

Un serveur qui va permettre aux jeunes de réaliser les missions et de les soumettre. Aux professeurs, ce site va permettre en plus de centraliser les travaux des jeunes, mais également de permettre d'ordonnancer les missions et de mettre des verrous sur les missions.

Nous souhaitons apporter une plate-forme facile à utiliser pour un professeur qui ne programme pas et pour des enfants pour apprendre la programmation. 

Nous souhaitons avoir une interface dans un navigateur qui est en lien avec un serveur sur lequel sont stockées les données.

%TODO missions