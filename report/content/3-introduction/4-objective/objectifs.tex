\section{Objectifs}
\label{intro-objectifs}
Le but de ce travail est de fournir une interface pour apprendre aux jeunes la programmation. Cette interface devra être dans un navigateur pour des questions de facilité d'utilisation. En effet, cette application est orientée vers les écoles et donc il faut une utilisation simple et intuitive faite pour des personnes n'ayant pas ou peu de connaissances en programmation et le minimum en informatique en général. L'intérêt d'une plateforme web est aussi de ne devoir rien installer sur les ordinateurs.

Il faut également deux types d'interface, une pour les jeunes où toutes les fonctions inutiles sont retirées ainsi que les options pour modifier l'interface. Une pour les professeurs pour leur permettre d'avoir accès aux fonctions qui permettent de cacher des blocs et des scripts.

Un serveur qui va permettre aux jeunes de réaliser les missions et de les soumettre. Aux professeurs, ce site va permettre en plus de centraliser les travaux des jeunes, mais également de permettre d'ordonnancer les missions et de mettre des verrous sur les missions.

Nous souhaitons apporter une plate-forme facile à utiliser pour un professeur qui ne programme pas et pour des enfants pour apprendre la programmation. 

Nous souhaitons avoir une interface dans un navigateur qui est en lien avec un serveur sur lequel sont stockées les données.
