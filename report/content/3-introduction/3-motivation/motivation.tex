\section{Motivation}
\label{intro-motivation}
L'éducation des jeunes à la programmation peut être bénéfique à la société sur de nombreux points comme la structuration logique de l'esprit, mais aussi une meilleure compréhension de l'environnement et un meilleur départ des enfants s'ils souhaitent partir dans l'informatique.

Ce mémoire a été sélectionné pour son côté concret dans sa finalité. Pour l'apport qu'il pourrait apporter et son impact direct qu'il pourrait avoir sur l'enseignement.
Un autre point important est l'interaction avec des enfants et le coté pédagogique.\\

L'Europe nous indique que la programmation est excellente pour la structuration logique de l'esprit apporté par la programmation. En effet, la programmation demande beaucoup de rigueur, une erreur ou un oubli dans la structuration ou dans la syntaxe entraine directement une erreur. L'algorithme qui peut être vu comme un enchainement de commande ou d'instruction est exactement l'exercice d'un raisonnement logique.

Dans les cours de la communauté française, il y a des cours de "technologie" qui ont des programmes très libres. Dans ces cours, plusieurs professeurs essayent d'apprendre l'informatique aux jeunes, mais par manque de supports, d'applications adéquates et probablement aussi par manque de connaissances, ils n'enseignent que trop peu la programmation et se contentent de bureautique. Rajouter la programmation aux cours de sciences serait aussi une autre bonne manière d'amener un outil d'analyse de données en plus de toutes les caractéristiques précitées de la programmation.

La technologie et l'informatique sont partout, pourtant bon nombre de personnes ne comprennent pas comment tout ça fonctionne. Il serait donc intéressant que les enfants ai une idée de comment fonctionnent les objets de leur quotidien.
