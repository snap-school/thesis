\section{Motivation}
\label{intro-motivation}
L'éducation des jeunes à la programmation peut être bénéfique à la société sur de nombreux points comme : la structuration logique de l'esprit, une meilleure compréhension de l'environnement ou encore un meilleur départ des enfants s'ils souhaitent se lancer dans l'informatique.

L'Europe nous indique que la programmation est un très bon exercice pour la structuration logique de l'esprit. En effet, la programmation demande beaucoup de rigueur, une erreur, un oubli dans la structuration ou dans la syntaxe entraine directement une erreur. L'algorithme vu comme un enchaînement de commande ou d'instruction est un exercice de raisonnement logique.

Dans les cours de technologie, par exemple, plusieurs professeurs essayent d'apprendre l'informatique aux jeunes. Mais par manque de supports, d'applications adéquates et probablement aussi par manque de connaissances dans ce domaine, ils n'enseignent que trop peu la programmation et se contentent de la bureautique. Ce travail peut fournir une aide précieuse pour tous ces professeurs en mettant à leur disposition une plateforme qui les soutiendrai dans leur enseignement.

La technologie et l'informatique sont partout, pourtant bon nombre de personnes ne comprennent pas comment tout ça fonctionne. Il serait donc intéressant que les enfants ai une idée de comment fonctionnent les objets de leur quotidien.
