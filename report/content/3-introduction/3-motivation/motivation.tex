\section{Motivation}
\label{intro-motivation}
La technologie et l'informatique sont partout, pourtant bon nombre de personnes ne comprennent pas comment cela fonctionne. Il est intéressant que les enfants aient une idée du fonctionnent des objets de leur quotidien.

L'éducation des jeunes à la programmation peut être bénéfique à la société sur de nombreux points comme : une structuration logique de l'esprit, une meilleure compréhension de l'environnement ou encore un meilleur départ pour enfants qui souhaitent se lancer dans l'informatique.

L'Europe \cite{rapport-europeen} nous indique que la programmation est un très bon exercice pour l'acquisition d'un raisonnement logique. En effet, la programmation demande beaucoup de rigueur. Une erreur, un oubli dans la structuration ou dans la syntaxe entraîne directement un bogue. %L'algorithme vu comme un enchaînement de commande ou d'instruction est un exercice de raisonnement logique.

Dans les cours de technologie, par exemple, plusieurs professeurs essayent d'apprendre l'informatique aux jeunes. Mais par manque de supports, d'applications adéquates et probablement aussi par manque de connaissances dans ce domaine, ils n'enseignent que trop peu la programmation et se contentent de la bureautique. Ce travail veut fournir une aide pour tous ces professeurs en mettant à leur disposition une plateforme qui les soutiendrait dans leur enseignement.


