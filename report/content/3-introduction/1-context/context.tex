\section{Contexte}
\label{intro-context}
Dans un monde où la pédagogie est un sujet de société et où l'informatique est omniprésente, on s'étonne de n'observer que peu d'interactions entre ces deux sujets. Il existe toutefois quelques recherches et initiatives à ce sujet. Il faut encore distinguer celles qui concernent l'informatique dans son ensemble et celles qui se portent sur la programmation.\\

Il existe deux écoles d'apprentissage de la programmation. La première prône l'apprentissage de la programmation directement avec des langages de programmation classique. D'autres préfèrent des langages simplifiés par exemple la programmation par blocs. 

Ces deux écoles se différencient essentiellement par rapport au public qu'elles visent. Une programmation par blocs sera plus simple dans le premier temps et donc sera plus adaptée aux enfants plus jeunes n'ayant aucune expérience avec la programmation. Une approche avec un langage complet comme python sera plus adaptée pour des enfants plus âgés ou ayant déjà une expérience.\\

Des initiative similaires sont disponible dans d'autre pays et d'autres langue. Mais aucune de ces initiative n'a été conçue pour un publique francophone.



%Dans le domaine de la recherche, nous retiendrons essentiellement une publication de la communauté européenne qui stipule que l'apprentissage de la programmation apporte un soutien a la structuration de raisonnement logique.\\ %TODO revoir la formulation pourquoi que l'europe?

%Nous clôturerons cette partie avec un constat, dans une société où l'informatique occupe une place prédominante, nous constatons que la connaissance de la programmation n'est pas encore beaucoup valorisée et stimulée.  %TODO ??
