\chapter{Introduction}
Ce travail présente un plateforme d'apprentissage de la programmation appelée \gls{rsnap}. Cette plateforme est destinée à soutenir l'apprentissage de la programmation chez les enfants de 10 à 14 ans.

Ce chapitre introductif commence avec une mise en contexte du problème suivit d'une brève description de celui-ci, vient ensuite les motivations qui ont incitées à la réalisation de ce mémoire. Les objectifs sont abordé ensuite, suivit de l'approche choisie. Ce chapitre se termine avec les contributions que ce travail apporte à l'état de l'art et le plan du mémoire.
 
\subsection{Context}
On va parler des programme pour apprendre a des jeunes la programmation par blocs
Notre monde est de plus en plus tourn� vers l'informatique mais peu savent programmer blablabla
\section{Problème}
\label{into-problem}
Comme introduit dans la section précédente, il y a un manque de soutien à l'apprentissage de la programmation pour les jeunes francophones. Il n'y a pas ou peu de moyen pédagogique actuellement. Ce manque de moyen n'incite pas ceux qui souhaiteraient prendre le train en marche et commencer à enseigner l'apprentissage de la programmation

Il y a donc un besoin pour une plateforme spécialisée dans l'apprentissage de la programmation accessible à tous. Il n'est pas réaliste de demander au professeur d'être programmeur. Il est donc nécessaire que cette plateforme soit utilisable sans avoir de connaissances particulières dans ce domaine.

\section{Motivation}
\label{intro-motivation}
L'éducation des jeunes à la programmation peut être bénéfique à la société sur de nombreux points comme la structuration logique de l'esprit, mais aussi une meilleure compréhension de l'environnement et un meilleur départ des enfants s'ils souhaitent se lancer dans l'informatique.

L'Europe nous indique que la programmation est excellente pour la structuration logique de l'esprit apporté par la programmation. En effet, la programmation demande beaucoup de rigueur, une erreur ou un oubli dans la structuration ou dans la syntaxe entraine directement une erreur. L'algorithme qui peut être vu comme un enchainement de commande ou d'instruction est exactement l'exercice d'un raisonnement logique.

Dans les cours de la communauté française, il y a des cours de "technologie" qui ont des programmes très libres. Dans ces cours, plusieurs professeurs essayent d'apprendre l'informatique aux jeunes, mais par manque de supports, d'applications adéquates et probablement aussi par manque de connaissances, ils n'enseignent que trop peu la programmation et se contentent de bureautique. Rajouter la programmation aux cours de sciences serait aussi une autre bonne manière d'amener un outil d'analyse de données en plus de toutes les caractéristiques précitées de la programmation. %TODO pas parler de techno

La technologie et l'informatique sont partout, pourtant bon nombre de personnes ne comprennent pas comment tout ça fonctionne. Il serait donc intéressant que les enfants ai une idée de comment fonctionnent les objets de leur quotidien.

Ce mémoire a été sélectionné pour son côté concret dans sa finalité. Pour l'apport qu'il pourrait apporter et son impact direct qu'il pourrait avoir sur l'enseignement.
Un autre point important est l'interaction avec des enfants et le coté pédagogique. %TODO a garder?


\section{Objectifs}
\label{intro-objectifs}

Deux objectif principaux sont poursuivit par ce travail:

\paragraph{Plateforme d'apprentissage} Fournir une plateforme qui permet d'enseigner la programmation aux jeunes de 10 à 14 ans. Cette plateforme doit être disponible partout et facile d'utilisation. Cette dernière doit également pouvoir stocker des informations pour qu'elles soient disponibles lors de prochaines utilisations.

\paragraph{Création de contenu} Fournir du contenu à la plateforme définie plus haut. A savoir des exercices qui ont pour but d'introduire la programmation chez les jeunes. Un contenu initial est très important pour servir d'exemple à la création de future contenu.\\

%TODO checker la mise en forme

Au delà de ces deux objectifs, un troisième a déjà été annoncé plus tôt dans ce rapport. Ce travail est à destination d'un publique francophone, il est donc primordial que la plateforme ainsi que le contenu soit en français.



%TODO abstraire
%Le but de ce travail est de fournir une interface pour apprendre aux jeunes la programmation. Cette interface devra être dans un navigateur pour des questions de facilité d'utilisation. En effet, cette application est orientée vers les écoles et donc il faut une utilisation simple et intuitive faite pour des personnes n'ayant pas ou peu de connaissances en programmation et le minimum en informatique en général. L'intérêt d'une plateforme web est aussi de ne devoir rien installer sur les ordinateurs.
%
%Il faut également deux types d'interface, une pour les jeunes où toutes les fonctions inutiles sont retirées ainsi que les options pour modifier l'interface. Une pour les professeurs pour leur permettre d'avoir accès aux fonctions qui permettent de cacher des blocs et des scripts.
%
%Un serveur qui va permettre aux jeunes de réaliser les missions et de les soumettre. Aux professeurs, ce site va permettre en plus de centraliser les travaux des jeunes, mais également de permettre d'ordonnancer les missions et de mettre des verrous sur les missions.
%
%Nous souhaitons apporter une plate-forme facile à utiliser pour un professeur qui ne programme pas et pour des enfants pour apprendre la programmation. 
%
%Nous souhaitons avoir une interface dans un navigateur qui est en lien avec un serveur sur lequel sont stockées les données.

%TODO missions
\section{Approche}
\label{intro-approche}
%TODO \gls{Rsnap}
L'application développée dans le cadre de ce travail est nommé \gls{rsnap}.

\paragraph{La plateforme} La plateforme sera implémenté par une combinaison entre un environnement de programmation existant \gls{snap} \cite{snap} et un site web en \gls{rails} \cite{rails}. Cette combinaison permet d'avoir une plateforme accessible par tous les ordinateurs ayant accès à internet et un navigateur récent.

La plateforme devant aussi permettre la création de contenu par les professeurs, il sera nécessaire d'introduire des \glspl{role} différent pour ceux-ci et les jeunes.

\paragraph{Le contenu} Le contenu initial fourni sera une série de \glspl{mission} qui se succède, la fin de l'une débloquant la suivante. Cette série d'exercices introduira les concepts nécessaire à la création d'un jeu plus conséquent. Ce jeu sera un jeu de poursuite permettant aux jeunes de jouer entre eux.

\section{Contributions}
\label{intro-contribution}
Papier sur les biens faits de l'apprentissage de la programmation sur l'esprit logique.

\paragraph{Un outil francophone} Comme introduit dans la section \ref{into-problem} il n'existait pas encore ou peu de matériel francophone pour soutenir l'apprentissage de la programmation. \gls{rsnap} est une plateforme entièrement francophone.%TODO pour moi on peut supprimer la suite : et les traductions de \gls{snap} on été finie et améliorée dans le cadre de ce travail.

\paragraph{Une plateforme d'apprentissage} Aucune plateforme n'est actuellement disponible pour l'apprentissage de la programmation. \gls{rsnap} se veut être orienté pour l'apprentissage dans l'enseignement traditionnel. Ceci en intégrant une gestion de classe, une progression contrôlable par l'enseignant. \gls{rsnap} à été pensée dès le début pour que les élèves ainsi que les proffesseurs puissent l'utiliser de la manière la plus autonome possible.

\section{Plan du mémoire}
La suite de ce mémoire se divise de la manière suivante:

\paragraph{Connaissances nécessaire}
Ce chapitre introduit le bagage technique nécessaire à la bonne compréhension de ce travail. Il commence par quelques prérequis, puis introduit Snap! et enfin parle plus en détail de Ruby on Rails.

\paragraph{Travaux associés}
Ce chapitre parle de l'état de l'art dans le monde. Il présente l'avancée de l'apprentissage de la programmation dans le monde. Ensuite, il présente quelques grandes initiatives similaires à celle de ce travail. Après quoi, il introduit les différents langages de programmations enseignés aux jeunes. Et se finit par une liste de concepts différenciateurs qui permet de définir une initiative de manière unique.

\paragraph{Définition de la problématique}
Ce chapitre présente la position de ce mémoire par rapport au chapitre précédent. Il positionne \gls{Rsnap} par rapport au concepts différenciateurs introduits dans les travaux associés. Et pour finir, il décrit les choix technologiques faits dans le cadre de ce travail.

\paragraph{Solution}
Ce chapitre décrit la plateforme \gls{Rsnap} implémentée dans le cadre de ce mémoire.
Il commence par décrire les missions qui ont été crées pour ce travail. Il parle ensuite des modifications apportées à Snap! pour les besoins du mémoire. Il fini par décrire la partie web de \gls{Rsnap}.

\paragraph{Validation}
Ce chapitre présente la validation de l'application. Il commence par les tests sur l'application \gls{Rsnap}. Il présente ensuite les expériences qui ont été menées. Et fini par l'analyse des formulaires proposé aux participants de l'expérience.

\paragraph{Conclusion}
Ce chapitre conclu ce travail, compare les objectifs attendus et atteints, les points plus faibles et les points fort de ce mémoire. %TODO à retravailler quand elle sera écrite

\paragraph{Futures amélioration}
Ce chapitre décrit ce qui serait utile de faire dans la continuité de ce travail. Ce sont des améliorations qui n'ont pas pues être réalisées soit par manque de temps soit parce qu'elles sont apparues dans l'analyse des expériences.

%Dans cette section seront présentés les prérequis et un état des lieux de l'apprentissage de la programmation aux enfants dans le monde. En premier lieu, il y aura les prérequis et SNAP BYOB. Ensuite un analyse d'autres initiatives similaires à celle de ce travail. Suite à cela vendra la position du projet \gls{Rsnap} par rapport à ce qui est déjà existant et expliqué dans les points précédents, suivis d'une présentation des outils utilisés. La dernière partie portera sur ce qu'apporte \gls{Rsnap} par rapport aux autres projets existants.

