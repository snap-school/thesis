\chapter{Introduction}
Ce travail présente un plateforme d'apprentissage de la programmation appelée \gls{rsnap}. Cette plateforme est destinée à soutenir l'apprentissage de la programmation chez les enfants de 10 à 14 ans.

Ce chapitre introductif commence avec une mise en contexte du problème suivit d'une brève description de celui-ci, vient ensuite les motivations qui ont incitées à la réalisation de ce mémoire. Les objectifs sont abordé ensuite, suivit de l'approche choisie. Ce chapitre se termine avec les contributions que ce travail apporte à l'état de l'art et le plan du mémoire.
 
\section{Contexte}
\label{intro-context}
Dans un monde où la pédagogie est un sujet de société et où l'informatique est omniprésente, on s'étonne de n'observer que peu d'interactions entre ces deux sujets. Il existe toutefois quelques recherches et initiatives. Il faut encore distinguer celles qui concernent l'informatique dans son ensemble et celles qui se portent sur la programmation.\\

Il existe deux écoles d'apprentissage de la programmation. La première prône l'apprentissage de la programmation directement avec des langages de programmation classique. D'autres préfèrent des langages simplifiés ; par exemple la programmation par \glspl{bloc}.

Ces deux écoles se différencient essentiellement par rapport au public qu'elles visent. Une programmation par \glspl{bloc} sera plus simple dans un premier temps et donc sera mieux adaptée aux enfants plus jeunes n'ayant aucune expérience avec la programmation. Une approche avec un langage complet comme Python sera plus adaptée pour des enfants plus âgés ou ayant déjà une expérience.\\

Des initiatives sont disponibles dans d'autres pays et d'autres langues. Mais aucune de ces initiatives n'a été conçue pour un public francophone.



%Dans le domaine de la recherche, nous retiendrons essentiellement une publication de la communauté européenne qui stipule que l'apprentissage de la programmation apporte un soutien a la structuration de raisonnement logique.\\ %TODO revoir la formulation pourquoi que l'Europe?

%Nous clôturerons cette partie avec un constat, dans une société où l'informatique occupe une place prédominante, nous constatons que la connaissance de la programmation n'est pas encore beaucoup valorisée et stimulée.  %TODO ??

\section{Problème}
\label{into-problem}
Le but de ce travail est de promouvoir l'apprentissage de la programmation auprès des plus jeunes. En effet là nous avons vu dans la section \ref{intro-context} qu'il y a un manque de moyens pédagogiques dans ce domaine. Ce travail propose une adaptation d'un programme informatique existant Snap! mais en amenant cela dans une interface web et soutenu par un serveur. Ce travail vise donc à produire une nouvelle plateforme spécialisée dans l'apprentissage de la programmation accessible à tous, aussi bien les professeurs que les élèves. L'enseignant pourra grâce à l'application donner des cours de programmation à ses classes sans avoir de connaissances particulières.

En plus de l'application, un ensemble de missions simple a été réalisé et testé sur des enfants pour montrer que ce projet est viable. Les missions proposées devaient avoir aussi bien des objectifs pédagogiques que ludiques.

\section{Motivation}
\label{intro-motivation}
L'éducation des jeunes à la programmation peut être bénéfique à la société sur de nombreux points comme : la structuration logique de l'esprit, une meilleure compréhension de l'environnement ou encore un meilleur départ des enfants s'ils souhaitent se lancer dans l'informatique.

L'Europe nous indique que la programmation est un très bon exercice pour la structuration logique de l'esprit. En effet, la programmation demande beaucoup de rigueur, une erreur, un oubli dans la structuration ou dans la syntaxe entraine directement une erreur. L'algorithme vu comme un enchaînement de commande ou d'instruction est un exercice de raisonnement logique.

Dans les cours de technologie, par exemple, plusieurs professeurs essayent d'apprendre l'informatique aux jeunes. Mais par manque de supports, d'applications adéquates et probablement aussi par manque de connaissances dans ce domaine, ils n'enseignent que trop peu la programmation et se contentent de la bureautique. Ce travail peut fournir une aide précieuse pour tous ces professeurs en mettant à leur disposition une plateforme qui les soutiendrai dans leur enseignement.

La technologie et l'informatique sont partout, pourtant bon nombre de personnes ne comprennent pas comment tout ça fonctionne. Il serait donc intéressant que les enfants ai une idée de comment fonctionnent les objets de leur quotidien.

\section{Objectifs}
\label{intro-objectifs}
Le but de ce travail est de fournir une interface pour apprendre aux jeunes la programmation. Cette interface devra être dans un navigateur pour des questions de facilité d'utilisation. En effet, cette application est orientée vers les écoles et donc il faut une utilisation simple et intuitive faite pour des personnes n'ayant pas ou peu de connaissances en programmation et le minimum en informatique en général. L'intérêt d'une plateforme web est aussi de ne devoir rien installer sur les ordinateurs.

Il faut également deux types d'interface, une pour les jeunes où toutes les fonctions inutiles sont retirées ainsi que les options pour modifier l'interface. Une pour les professeurs pour leur permettre d'avoir accès aux fonctions qui permettent de cacher des blocs et des scripts.

Un serveur qui va permettre aux jeunes de réaliser les missions et de les soumettre. Aux professeurs, ce site va permettre en plus de centraliser les travaux des jeunes, mais également de permettre d'ordonnancer les missions et de mettre des verrous sur les missions.

Nous souhaitons apporter une plate-forme facile à utiliser pour un professeur qui ne programme pas et pour des enfants pour apprendre la programmation. 

Nous souhaitons avoir une interface dans un navigateur qui est en lien avec un serveur sur lequel sont stockées les données.

\section{Approche}
\label{intro-approche}
%TODO \gls{Rsnap}
L'application développée dans le cadre de ce travail est nommée \gls{rsnap}.

\paragraph{La plateforme} La plateforme sera implémentée par une combinaison entre un environnement de programmation existant \gls{snap} \cite{snap} et un site web en \gls{rails} \cite{rails}. Cette combinaison permet d'avoir une plateforme accessible par tous les ordinateurs ayant accès à internet et un navigateur récent.

La plateforme devant aussi permettre la création de contenu par les professeurs, il sera nécessaire d'introduire des \glspl{role} différent pour ceux-ci et les jeunes.

\paragraph{Le contenu} Le contenu initial fourni sera une série de \glspl{mission} qui se succède, la fin de l'une débloquant la suivante. Cette série d'exercices introduira les concepts nécessaires à la création d'un jeu plus conséquent. Ce jeu sera un jeu de poursuite permettant aux jeunes de jouer entre eux.

\section{Contributions}
\label{intro-contribution}
Cette section présente les différentes contribution de ce travail à l'existant. Nous retiendront principalement: un outils francophone pour apprendre la programmation, une plateforme intégrée orienté pour l'enseignement et la combinaison d'outils existant.

\paragraph{Un outil francophone} Comme introduit dans la section \ref{into-problem} il n'existait pas encore ou peu de matériel francophone pour soutenir l'apprentissage de la programmation. Rsnap est une plateforme entièrement francophone et les traductions de Snap! on été finie et améliorée dans le cadre de ce travail.

\paragraph{Une plateforme utilisable} Aucune plateforme n'est actuellement disponible pour l'apprentissage de la programmation. Rsnap se veut être orienté pour l'apprentissage dans l'enseignement traditionnel. Ceci en intégrant une gestion de classe, une progression contrôlable par l'enseignant. Rsnap à été pensée dès le début pour que les élèves puissent l'utilisé de la manière la plus autonome possible.

\paragraph{Combinaison d'outils existant} Dans le cadre de ce mémoire les ressources tant au niveau temps que matérielle sont comptée. Ce travail montre également que à l'aide de plateforme libre existante et de technologie tel que JavaScript ou encore Ruby on Rails, il est possible de crée une plateforme d'apprentissage aboutie.




%
%
%
%
%%TODO Programmation francophone, plateforme, montrer que c'est facile qu'il y a deja plein d'outils, 
%Papier sur les biens faits de l'apprentissage de la programmation sur l'esprit logique.
%
%Le projet SNAP est à l'origine maintenu par jmoening. C'est la principale contribution au projet. Dans ce travail, certaines parties ont pu être intégrées au projet, par exemple la traduction des aides en français a pu être ajoutée au projet de jmoening. Deux autres pull request sont en court. La première concerne l'ajout de rôle pour pouvoir modifier l'interface facilement en fonction des rôles que l'on souhaite avoir. La seconde porte sur une traduction automatisée de l'interface. Une autre pourrait encore être proposée pour les fonctions d'importation et d'exportation qui ont été ajoutées pour ce travail.
%
%Jimoegning bref lui :)
%
%Tous les autres intervenants
%
%Écoles ?
%
%Printemps des sciences ?

\section{Plan du mémoire}
La suite de ce mémoire se divise de la manière suivante:

\paragraph{Connaissances nécessaires}
Ce chapitre introduit le bagage technique nécessaire à la bonne compréhension de ce travail. Il commence par quelques prérequis, puis introduit \gls{snap} et enfin parle plus en détail de \gls{rails}.

\paragraph{Travaux associés}
Ce chapitre parle de l'état de l'art dans le monde. Il présente l'avancée de l'apprentissage de la programmation dans le monde. Ensuite, il présente quelques grandes initiatives similaires à celle de ce travail. Après quoi, il introduit les différents langages de programmation enseignés aux jeunes. Et se finit par une liste de concepts différenciateurs qui permet de définir une initiative de manière unique.

\paragraph{Définition de la problématique}
Ce chapitre présente la position de ce mémoire par rapport au chapitre précédent. Il positionne \gls{rsnap} par rapport aux concepts différenciateurs introduits dans les travaux associés. Et pour finir, il décrit les choix technologiques faits dans le cadre de ce travail.

\paragraph{Solution}
Ce chapitre décrit la plateforme \gls{rsnap} implémentée dans le cadre de ce mémoire.
Il commence par décrire les \glspl{mission} qui ont été crées pour ce travail. Il parle ensuite des modifications apportées à \gls{snap} pour les besoins du mémoire. Il fini par décrire la partie web de \gls{rsnap}.

\paragraph{Validation}
Ce chapitre présente la validation de l'application. Il commence par les tests sur l'application \gls{rsnap}. Il présente ensuite les expériences qui ont été menées. Et fini par l'analyse des formulaires proposée aux participants de l'expérience.

\paragraph{Conclusion}
Ce chapitre conclut ce travail, compare les objectifs attendus et atteints, les points plus faibles et les points forts de ce mémoire. %TODO à retravailler quand elle sera écrite

\paragraph{Futures amélioration}
Ce chapitre décrit ce qui serait utile de faire dans la continuité de ce travail. Ce sont des améliorations qui n'ont pas pu être réalisées soit par manque de temps soit parce qu'elles sont apparues dans l'analyse des expériences.

