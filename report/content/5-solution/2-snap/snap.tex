\section{SNAP}
Comme expliqué précédemment Nous somme parti d'un projet existant, SNAP BYOB, pour l'environnement de programmation. Ce projet a pour but de fournir une interface et un environnement supportant la programmation par bloc. Ce projet ne s'inscrit pas dans le cadre d'un apprentissage scolaire ou guidée. Il a donc fallu adapter le projet pour une utilisation plus scolaire. Nous allons expliquer dans cette partie les différentes adaptations que nous avons apporter au projet d'origine et pourquoi elles sont nécessaire pour remplir nos objectifs.\\

Dans les adaptations que nous avons du opéré nous retiendront: une simplification de l'interface, des fonctionnalité suplémentaire tel que la sauvegarde sur les serveur du projet courant, une différenciation de rôle pour le professeur par rapport au public cible, un amélioration de la traduction en francais.

\subsection{L'interface}
Comme expliqué pécédemment le projet original n'avait pas de vocation didactique de groupe et avait un publique cible plus large que le notre et plus agé. Dans cette idée il y avait  des menus pour des fonctionnalité de gestion du scheduler et autre. Il est évident que ces fonctionnalité ne sont pas utile pour notre publique. De plus vu l'age de notre public toutes distractions qui peut être évité améliore sensiblement la concentration de ces dernier.

Dans cet optique nous avons opéré un néttoyage en profondeur de l'interface dans l'optique de laisser uniquement les menu utile. Toute fois comme il sera discuter dans les rôles il était interessant de ne pas simplement les effacer mais bien de les masquer.

\subsection{Fonctionnalité suplémentaire}
Comme nous avons interfacer l'application de programmation avec un site web, nous avons du reimplementer certaines fonctionnalité d'import export.\\

Nous avions besoin de pouvoir passer le projet à l'ouverture de l'application. Pour cela la majorité des fonctions était déjà présente. La technique utiliser était de passer l'XML contenant le projet dans la barre d'adresse du navigateur. Quelques adaptation on permis de masquer le passage du projet a l'utilisateur.\\

Dans les fonctions d'export il y avait ici aussi des solution existante mais encore une fois, n'étant pas la priorité des personnes maintenant le projet, ces fonctions étaient peu pratique. Lors d'un export de projet, le fichier xml généré était généré dans une page html. Nous avons repris ces fonctions afin de créer le fichier html dans les temps ensuite de l'envoyer sur le serveur à l'aide d'une requet \texttt{push}

tout les changement de l'interface

bouton d'aide

roles

export import

traduction ???

aides
