\chapter{Future améliorations}
Cette section va aborder les amélioration future possible pour Rsnap. Ces dernières sont soit issus d'objectif qui n'ont pas été rencontré par l'implémentation actuelle soit de l'analyse de l'expérience. Les améliorations suivant sont a retenir et sont discutées plus dans ce chapitre: test de la réussite de la mission, aide à la gestion de la classe, retravail des missions, ajout de mission, blocs d'aide a la correction automatique des programme.

\section{Test de réussite de mission}
Actuellement certain tests sont réaliser pour la réussite de la mission. Toute fois, aucun signale de réussite n'est envoyer vers les serveurs. Une mission est considéré comme effectuer dès que l'étudiant ouvre la mission. En effet, comme aucun méchanisme n'est présent pour validé une mission, la mission suivante se débloque dès l'ouverture de la précédente.\\

Un signal envoyer au serveur serai possible par le simple envoie d'une requête à ce dernier. La difficulté réside dans la création d'un vrai bloc disponible dans l'interface \texttt{teacher} pour notifier au serveur la réussite d'une mission.

\section{Aide à la gestion des classes}
Un des but de ce travail est son utilisation dans l'enseignement. Pour ce faire un système de gestion des élèves par classe a été pensé. Malheureusement, le temps a manqué pour avoir une gestion de classe finie. 

Le principe pensé pour la gestion de classe est que chaque professeurs puisse voir l'avancement de ses élèves. Pour ce faire il faut que les élèves désigne un professeur dans leur profile.

\section{Retravail des missions}
Comme dit dans la partie \ref



obliger la réussite des test de la mission avant de pouvoir passer à la suivante

gestion des classes

rearanger les missions voir analyse

plus de split dans les missions

plus de missions

block de correction automatique des missions

charte graphique plus cohérente a travers le site et snap! et sur les présentation des missions.
