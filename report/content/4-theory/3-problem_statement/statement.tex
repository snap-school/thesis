\section{Définition de la problématique}
A la lumière des pratiques dans les autres pays, des concepts différenciateurs, du positionnement de Rsnap vis à vis des ces concepts et des caractéristiques plus techniques de SNAP! et Rails. Cette partie est dédiée à explication précise de la problématique aux apports de ce travail à l'état de l'art.

Comme développé dans le chapitre \ref{monde}, l'apprentissage de la programmation est déjà bien avancé dans plusieurs pays. Ce n'est malheureusement pas encore tout à fait le cas dans le notre. En effet quelques initiatives locales existe mais aucune décision politique n'a été prise jusqu'à présent. Le but de ce travail est de fournir une plateforme qui permette cet apprentissage.\\

En effet, à l'heure actuelle, aucune plateforme n'est disponible en français et de manière plus générale peu dans le monde se veulent orienté vers les professeurs. C'est ces lacunes que Rsnap souhaite palier. Sur le plan du langage, la plate forme Rsnap se veut complètement en français pour pouvoir être utiliser par tous les enfants inscrit dans l'enseignement francophone sachant lire. Sur l'orientation professeur imprimée dans Rsnap, se traduit par une gestion de classe, une indépendance des enfants par rapport à un référent, l'absence de prérequis pour le professeur et l'édition de missions qui est communautaire.\\


%fait le 18/4
1e partie (apprentissage dans le monde) qu'apporte notre solution : une interface prof, enfants francophones

2e partie (technologie) toutes les briques sont déjà dispo pour fournir un apprentissage au enfants, il ne reste qu'a mettre un peu de glue pour tout assembler.
