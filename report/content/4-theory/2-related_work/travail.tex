\section{Travail associé}

\subsection{Code.org}
http://code.org/about
Code.org est une organisation sans but lucratif des USA qui à pour objectif :
\begin{itemize}
  \item apporter l'informatique dans toutes les classes de secondaire des usa.
  \item démontrer le succes de l'utilisation de cours en ligne dans l'ensseignement public
  \item Changer les bases des programmes de sciences/math des 50 états pour faire entrer l'informatique dedans
  \item Employé la connaisance technique collective pour améliorer l'apprentissage de l'informatique dans le monde
  \item Augmenter la représentation des femmes et des personnes de couleur en informatique
\end{itemize}

http://code.org/educate/20hr
pour ce faire, il fournisse une platforme web qui permet à des prof de mettre en place des classes pour suivre l'évolution de leur élèves.

toutes les ressources sont gratuites et librement utilisable (http://code.org/faq). Leur programme d'apprentissage se base sur Blockly (voir plus bas).
Les ressources sont concue pour que aussi bien les prof que les étudants puissent commencer le cours sans connaitre l'informatique (un assistance est proposé au prof si nécessaire gratuitement).

Il propose aux profs de se faire récompenser si il arrive à faire terminer à 15 étudiants les 27 missions proposée (750\$) et si ils ont au moins 7 filles (250\$)

\subsubsection{déroulement des lecons}
Des videos d'introductions son présentée pour \textbf{inspirer les élèves}, leur montrer tout ce que la programation permet

propose de faire travailler les etudiant par pair (\textbf{pair programming}). ca permet d'avoir moins de questions pour le prof et de mieux s'approprier la matière. Permet de casser l'image tu geek en montrant que la programmation est une sciences : sociale et colaborative. et moins d'ordi nécessaire

\textbf{faire participer tout les élèves} en faisant confiance en leur compétance : permettre au premier groupe d'aider les derniers.

Pour résoudre un problème, il est recommander de proposer au étudiants de \textbf{d'abord demander à 3 de leur camarades} avant de poser la question au prof. Le prof ne dois pas non plus etre compétent, il doit juste pouvoir réfléchir avec les élèves de quel est le problème.

\subsubsection{Projet sur internet}
La métodologie utiliser par code.org est la suivante:

Il propose des session de une heure de travail/jeu/apprentissage. Chaque unité d'une heure est découper en petite mission (ex:5-20) les missions sont tres courte et apporte un concepte de programmation. Avant l'introduction de chaque concept une petite vidéo est faire pour expliquer le concept introduit.

Lorsque la personne est arrivée a la fin d'une unité (une heure) elle recoit un trophé. Il y a 27 trophé au total. Ceci permet de bien visualiser sa progression et est motivant.

Pour chaque petite mission il y a un test automatisé qui dit si la mission est réussie ou non. Si la mission est réussie, le programme passe à la mission suivante. Il y a également un compteur de blocs dans les premiere mission. Ce compteur permet de voir combien de bloc sont nécessaire pour réaliser la mission de manière optimale.

Quand on rate la mission il y a des aides qui viennent pour guider les moins doué

\subsection{Blockly}
https://code.google.com/p/blockly/

c'est un éditeur de programation graphique basé sur des techno web. 

https://code.google.com/p/blockly/wiki/Alternatives
il est influencé par "App Inventor" qui est influencé par "Scratch" qui lui mm est influencé par "StarLogo"

Il permet de s'executer dans un navigateur ; d'exporter du code source js, dart... ; open source ; haut-niveau.
Il n'est pas directement une platforme d'éducation dans le sens ou suivant les blocs implémenté, il sera utilisé pour l'education, programmer du buisnes, des jeux ...

https://code.google.com/p/blockly/wiki/Language
language design philosophy de blockly :

[1] :facilite la compréhension par monsieur tout le monde
\begin{itemize}
  \item doit pouvoir faire un export js
  \item index des list commance à 1 [1]
  \item nom des variables non sensible à la casse [1]
  \item pas de scope de variable (toutes globale) [1]
  \item code natif généré proche de celui des bloques (meme si ce n'est pas optimal)
\end{itemize}

\section{CoderDojo}
http://coderdojo.com/about
CoderDojo est un réseau open source de club de programmation dans le sens le plus large du terme. Tous les dojos sont donc autonome.  Dans les dojos, des enfant de 5 à 17 ans apprennent la programmation (site web, application, jeux...). la seul règle est  “Above All: Be Cool“ qui peut etre mise en pratique simplement en créant des espace d'échanges de savoir sympa et sociable.

CoderDojo à été crée par James Whelton un irlandais de 18ans et Bill Liao un entrepreneur australien à Cork. James a eu des demandes de jeunes enfants pour avoir des cours de programmation après que James eut hacké l'ipod nano. Pas mal de gens de Dublin vinrent à ses cours et donc un nouveau Dojo à été crée à Dublin et puis partout dans le monde.

\url{http://coderdojo.com/news/2014/03/24/coderdojo-co-founder-james-whelton-talks-guardian}
Angy bird à été crée par un ancien étudiant de CoderDojo

\section{Code Club}
\url{https://www.codeclub.org.uk/about}
Code Club is a nationwide network of free volunteer-led after-school coding clubs for children aged 9-11.
Ils créent donc le matériel pour permettre à des bénévole de donner des cours parascolaire d'environ une heure semaine. Il propose dans l'ordre d'utiliser scratch, html/css et python. ils aimeraient que les 21 000 écoles primaires anglaises ai leur club.

Leur philosophie est de d'abord l'amusement, la créativité et l'exploration avant l'apprentissage des concept de programmation.

\section{L'état de la programmation en Europe}
Nous allons ici faire un tour d'horizon de différent pays d'européen ou non qui enseignent la programmation aux jeunes. 
\subsection{England}
\url{https://www.gov.uk/government/collections/statutory-guidance-schools#national-curriculum-from-september-2014}
L'apprentissage de l'informatique en Angleterre n'est pas nouveau. Pendant longtemps cet apprentissage était centré sur les technologies de l'information et de la communication (TIC). En 2010 un étude a été commandée à la Royal Society pour évaluer cet apprentissage. Un an plus tard leur rapport a révélé que l'enseignement tel que dispensé jusque là, n'était ni efficace ni en adéquation avec l'évolution de l'informatique dans notre société. La Royal Society suggère de changer les matières abordées en informatique. En effet précédemment ce sont les TIC qui étaient prescrites. L'apprentissage de la programmation serai plus bénéfique et adapter pour les enfants. Sur base de ce rapport les programmes de cours ont été adaptés.


\subsection{France}
\url{http://fr.wikipedia.org/wiki/Informatique\_et\_sciences\_du\_num\%C3\%A9rique}
\url{http://fr.wikipedia.org/wiki/Baccalaur\%C3\%A9at\_scientifique}
En France depuis deux an l'informatique fait partie intégrante du programme du baccalauréat de type S. Une des matière dispensée est "Informatique et sciences du numérique". Cette matière se subdivise en quatre sous parties qui sont : représentation de l'information, algorithmique, langage et programmation, architectures matérielles. Cette approche est donc également basé sur l'apprentissage de la programmation plutôt que sur les TIC.

\subsection{Nouvelle Zélande}
Ce pays à adopté récemment les sciences informatique dans sont programme d'étude. Les cours sont dispenser à partir de 15 ans. Les cours dispensé concerne l'apprentissage de la programmation et des concepts informatique en général. Nous avons un choix un peu différent dans ce pays sur l'âge du début de l'apprentissage. Beaucoup de pays commence plus jeune et introduise des concepts basiques. La Nouvelle Zélande s'inscrit dans une logique plus similaire à la France mais ne restreint pas l'informatique aux options scientifiques.

\subsection{Corée du Sud}
Enseigne l'informatique depuis longtemps et à tous les niveaux de l'enseignement. La culture numérique dans ces pays y est fort différent que par chez nous. Par exemple une carrière dans le gaming y est tout à fait normal. L'informatique est vraiment omni-présent dans ces cultures, il est donc normal que son apprentissage commence en primaire.

\subsection{Grèce}
L'apprentissage des sciences informatiques prend une place important dans les programme grec. Des 6 ans les enfants sont confronté à l'informatique à l'école. A cette age c'est plus de la maîtrise de l'outil qu'il apprennent. Dès 10 ans, leurs cours d'informatique prend une tournure plus algorithmique et donc plus proche de la computer sciences.

