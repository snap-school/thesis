\section{Travail associé}

\subsection{Code.org}
http://code.org/about
Code.org est une organisation sans but lucratif des USA qui à pour objectif :
\begin{itemize}
  \item apporter l'informatique dans toutes les classes de secondaire des usa.
  \item démontrer le succes de l'utilisation de cours en ligne dans l'ensseignement public
  \item Changer les bases des programmes de sciences/math des 50 états pour faire entrer l'informatique dedans
  \item Employé la connaisance technique collective pour améliorer l'apprentissage de l'informatique dans le monde
  \item Augmenter la représentation des femmes et des personnes de couleur en informatique
\end{itemize}

http://code.org/educate/20hr
pour ce faire, il fournisse une platforme web qui permet à des prof de mettre en place des classes pour suivre l'évolution de leur élèves.

toutes les ressources sont gratuites et librement utilisable (http://code.org/faq). Leur programme d'apprentissage se base sur Blockly (voir plus bas).
Les ressources sont concue pour que aussi bien les prof que les étudants puissent commencer le cours sans connaitre l'informatique (un assistance est proposé au prof si nécessaire gratuitement).

Il propose aux profs de se faire récompenser si il arrive à faire terminer à 15 étudiants les 27 missions proposée (750\$) et si ils ont au moins 7 filles (250\$)

\subsubsection{déroulement des lecons}
Des videos d'introductions son présentée pour \textbf{inspirer les élèves}, leur montrer tout ce que la programation permet

propose de faire travailler les etudiant par pair (\textbf{pair programming}). ca permet d'avoir moins de questions pour le prof et de mieux s'approprier la matière. Permet de casser l'image tu geek en montrant que la programmation est une sciences : sociale et colaborative. et moins d'ordi nécessaire

\textbf{faire participer tout les élèves} en faisant confiance en leur compétance : permettre au premier groupe d'aider les derniers.

Pour résoudre un problème, il est recommander de proposer au étudiants de \textbf{d'abord demander à 3 de leur camarades} avant de poser la question au prof. Le prof ne dois pas non plus etre compétent, il doit juste pouvoir réfléchir avec les élèves de quel est le problème.

\subsection{Blockly}
https://code.google.com/p/blockly/

c'est un éditeur de programation graphique basé sur des techno web. 

https://code.google.com/p/blockly/wiki/Alternatives
il est influencé par "App Inventor" qui est influencé par "Scratch" qui lui mm est influencé par "StarLogo"

Il permet de s'executer dans un navigateur ; d'exporter du code source js, dart... ; open source ; haut-niveau.
Il n'est pas directement une platforme d'éducation dans le sens ou suivant les blocs implémenté, il sera utilisé pour l'education, programmer du buisnes, des jeux ...

https://code.google.com/p/blockly/wiki/Language
language design philosophy de blockly :

[1] :facilite la compréhension par monsieur tout le monde
\begin{itemize}
  \item doit pouvoir faire un export js
  \item index des list commance à 1 [1]
  \item nom des variables non sensible à la casse [1]
  \item pas de scope de variable (toutes globale) [1]
  \item code natif généré proche de celui des bloques (meme si ce n'est pas optimal)
\end{itemize}

\section{CoderDojo}
http://coderdojo.com/about
CoderDojo est un réseau open source de club de programmation dans le sens le plus large du terme. Tous les dojos sont donc autonome.  Dans les dojos, des enfant de 5 à 17 ans apprenent la programmation (site web, application, jeux...). la seul règle est  “Above All: Be Cool“ qui peut etre mise en pratique simplement en créant des espace d'échanges de savoir simpa et sociable.

CoderDojo à été crée par James Whelton un irlandais de 18ans et Bill Liao un entrepreneur australien à Cork. James à eux des demandes de jeunes enfants pour avoir des cours de programmation après que James eux hacké l'ipod nano. Pas mal de gens de Dublin vienrent à ses cours et donc un nouveau Dojo à été crée à Dublin et puis partout dans le monde.

http://coderdojo.com/news/2014/03/24/coderdojo-co-founder-james-whelton-talks-guardian
Angy bird à été crée par un ancien etudiant de CoderDojo

\section{Code Club}
https://www.codeclub.org.uk/about
Code Club is a nationwide network of free volunteer-led after-school coding clubs for children aged 9-11.
Ils créent donc le materiel pour permettre à des bénévole de donner des cours parascolaire d'environ une heure semaine. Il propose dans l'orde d'utiliser scratch, html/css et python. ils aimeraient que les 21 000 écoles primaires anglaises ai leur club.

Leur philosophie est de d'abord l'amusement, la créativité et l'exploration avant l'apprensisage des concept de programmation.
