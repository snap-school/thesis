\section{Missions}
\label{missions}
Le moyen de présenter les concepts retenus est une approche par \glspl{mission}. Cette approche a été choisie sur base de l'analyse des autres initiatives similaires dans le chapitre \ref{travail-associe}. D'autres références présentant des cours avec Scratch ont aussi été utilisées tel que \cite{starting-from-scratch}, \cite{scratch-for-teens}, \cite{shall-we-learn}, \cite{pensee-info} et \cite{scratched}. Ces références ont été utilisé pour connaître le type d'exercices proposés aux enfants et s'en inspirer. Dans le cadre de ce travail, une série de \glspl{mission} ont été produites pour avoir une plateforme utilisable.

Après l'explication du schéma général des \glspl{mission}, une description de chacune des quatre \glspl{mission} est présentée ainsi que les concepts qu'elles apportent.

\subsection{Schéma général}
Une fois le choix de l'approche par mission fait, il a fallu définir comment amener la théorie de la programmation à travers ces \glspl{mission}. Il s'est avéré que pour capter l'attention et l'intérêt des élèves, il était nécessaire qu'elles aient un but final. Chaque \gls{mission} réussie mène à la mission suivante et introduit de nouveaux concepts de programmation. La dernière mission intègre les différents concepts à travers un jeu de poursuite. Dans ce jeu, un chien doit courir après un chat et le manger. Cet objectif ludique permet d'introduire plusieurs concepts de programmation intéressants tels que la succession d'instructions, les conditions, les boucles, les événements etc.

Les \glspl{mission} ont donc été conçues dans le but de pouvoir réaliser ce jeu, ce qui met l'accent d'avantage sur la réalisation d'un jeu que sur l'apprentissage de concepts. Dans cette optique, les concepts nécessaires à la réalisation de la \gls{mission} "Tu ne m'atterras pas" sont introduits par trois pré-missions. L'introduction progressive des concepts permet de se concentrer sur un à deux grands concepts par \gls{mission}. Ceci diminue les introductions théoriques des \glspl{mission} et augmente les temps pendant lesquels les enfants créent les programmes.

La chronologie des \glspl{mission} est pensée pour avoir une progression. La première introduit la programmation impérative, la seconde amène les boucles et la troisième donne l'intuition de la programmation événementielle. La dernière \gls{mission} réutilise tous ces concepts.

Les trois pré-missions sont : "en voiture", "l'hélicoptère" et "soyons courtois".


% Une fois le choix de l'approche par mission choisie, il faut définir comment amener la théorie de la programmation à travers ces missions. Il s'est avéré que pour capter l'attention et l'intérêt des élèves, il était nécessaire d'introduire les missions par la présentation d'un but final. Celui retenu dans ce travail a été le jeu du chien et du chat. Dans ce jeu le chien doit courir après le chat et le manger. Le principe de ce jeu est très simple et permet l'introduction de plusieurs concepts de programmation intéressants tels que la condition, les boucles et également les événements.\\
%
% Pour introduire les concepts de manière douce, la mission finale \texttt{Chien et chat} est introduite par trois missions d'introduction : \texttt{la voiture}, \texttt{l'hélicoptère} et \texttt{Soyons courtois}. Dans la suite de cette partie, nous allons décrire les missions et faire une analyse des concepts introduits.

\subsection{En voiture} \label{mission-voiture}
\begin{figure}[H]
  \begin{center}
    \includegraphics[width=\textwidth]{content/7-solution/1-missions/images/voiture}
    \caption{Mission de la voiture}
    \label{fig:mission-voiture}
  \end{center}
\end{figure}

Dans cette première \gls{mission}, voir figure \ref{fig:mission-voiture}, les participants sont mis dans un bolide qui doit atteindre la ligne d'arrivée en restant sur la route. Si la voiture sort de la route, elle explose. À chaque lancement du \gls{script} la voiture reprend sa position de départ.\\

Cette \gls{mission} vise à ce que les participants puissent prendre en main l'interface de \gls{snap} et à introduire la notion de suite d'instructions. Ils doivent enchaîner des \glspl{bloc} de déplacement dans lesquelles, le nombre de pas et l'angle de virage sont à compléter. A la base les élèves n'ont rien en dessous des premiers blocs jaunes, La figure \ref{fig:mission-voiture} montre la solution de la mission.

Le fait de devoir prévoir les instructions n'est pas un concept facile pour le public cible. Beaucoup exécutent une opération et puis cherchent à savoir comment continuer à partir de leur nouvelle position. Dans cette \gls{mission}, la voiture retournant à chaque fois à son point de départ, les élèves sont forcés à réfléchir de manière globale et anticipativement. Cette \gls{mission} les introduit également à la programmation impérative.\\

Cette \gls{mission} réussie, l'élève peut passer à la \gls{mission} de l'hélicoptère.


\subsection{L'hélicoptère} \label{mission-helicoptere}
\begin{figure}[H]
  \begin{center}
    \includegraphics[width=\textwidth]{content/7-solution/1-missions/images/helicoptere}
    \caption{Mission de l'hélicoptère}
    \label{fig:mission-hélicoptère}
  \end{center}
\end{figure}

Pour cette \gls{mission}, voir figure \ref{fig:mission-hélicoptère}, les élèves sont aux commandes d'un hélicoptère. Leur but est de faire le tour de la piste. La piste est un ellipsoïde bordé d'herbe. Cette forme ellipsoïde les force à utiliser des boucles et non un grand nombre de \glspl{bloc} créant une solution particulière. Si l'hélicoptère touche l'herbe, il explose. Sur le contour extérieur de la piste, une bande rouge et une bande bleue sont dessinées. Le but de cette \gls{mission} est de tourner dés qu'une de ces deux couleurs est touchée pour éviter l'explosion de l'hélicoptère. Comme pour la \gls{mission} précédente, à chaque lancement de \gls{script}, l'hélicoptère se repositionne à son point de départ. Au départ de la mission il n'y a que les blocs de tête qui sont présents.\\

Les concepts introduits par cette \gls{mission} sont :
\begin{itemize}
\item les boucles, particulièrement le \texttt{boucler infiniment} ;
\item la gestion des collisions grâce aux capteurs de couleurs ;
\item la division en sous-processus (un pour avancer, un pour la collision rouge et un pour la bleue).
\end{itemize}

\subsection{Soyons courtois} \label{mission-courtois}
\begin{figure}[H]
  \begin{center}
    \includegraphics[width=\textwidth]{content/7-solution/1-missions/images/courtois}
    \caption{Mission de soyons courtois}
    \label{fig:courtois}
  \end{center}
\end{figure}

Dans cette dernière \gls{mission} de préparation, voir figure \ref{fig:courtois}, le participant dirige un personnage qui croise d'autres personnages. Ces derniers disent bonjour à chaque fois qu'ils croisent quelqu'un. Le but de la \gls{mission} est que le personnage de l'élève réponde. Au départ de la mission il n'y a que les blocs de tête qui sont présents.

Les concepts introduits par cette \gls{mission} sont :
\begin{itemize}
\item la gestion des collisions grâce au capteur sur leur personnage ;
\item la division en sous-processus (un pour chaque personne) ;
\item l'introduction à l'interactivité de l'interface, leur personnage est déplaçable à l'aide des flèches du clavier ;
\item la gestion des dialogues et de l'affichage de texte.
\end{itemize}


\subsection{Tu ne m'attraperas jamais}
\label{chien-chat}
Cette \gls{mission} est la dernière de la série et a pour but de mettre ensemble tous les concepts vus dans les \glspl{mission} de préparation. Dans cette \gls{mission}, les élèves doivent faire en sorte d'avoir deux personnages et de pouvoir les déplacer séparément à l'aide des touches. Le chien doit courir après le chat. Sur base de propositions faites, les participants déterminent librement l'application exacte des effets de la poursuite. Par exemple, si le chat est touché par le chien, il crie. Aucun bloc n'est présent au départ de cette mission.\\

Cette \gls{mission} reprend les concepts vus dans les \glspl{mission} de préparation, auxquels elle ajoute :
\begin{itemize}
\item modification de l'interface de jeu ;
\item l'activation d'un \gls{script} sur un événement (déplacements).
\end{itemize}

En ce qui concerne les déplacements, ils sont déjà présents dans la \gls{mission} précédente, mais à l'état passif. Les élèves ne font que les utiliser. Dans cette \gls{mission}, ils doivent coder eux-mêmes le fait de pouvoir déplacer leurs personnages à l'aide du clavier. Cette application approfondie du concept de déplacement fixe un concept important pour utiliser une interaction avec le clavier.

%analyse mission voiture, cela à permis de prendre en main les entrées des \glspl{bloc}. Au début, beaucoup prennent 3 \glspl{bloc} avancer de 10 pas pour faire 30 pas au lieu de changer la valeur du 10 en 30.
