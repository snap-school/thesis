\subsection{Positionnement de \gls{Rsnap}}
Cette partie a pour but de positionner \gls{Rsnap} au travers des concepts mis en lumière par le chapitre précédent.

\subsubsection{Âge, origine et genre}
\gls{Rsnap} est une application qui vise un public de 10 à 14 ans. Les missions qui ont été développées dans le cadre de ce travail ciblent cette tranche d'âge particulièrement. Cette tranche sera nuancée lors de l'analyse des résultats de l'expérimentation sur les différentes tranches d'âge (voir \ref{trancheage}). %TODO faire la référence quand on écrira les tranches d'âge
toute foi l'application permettant la création de mission de manière autonome, il n'est pas exclu de créer des missions pour d'autres tranches d'âge. La créativité des utilisateurs est la seule limite formelle.\\

Quant aux origines et au genre, aucune discrimination dans quelque sens que ce soit n'a été faite lors de la réalisation de ce travail. Tant pour attirer un public peu représenté que pour l'exclure. Hormis le fait que l'application a été pensée pour un public francophone pour des raisons évidentes, c'est une problématique qui n'a tout simplement pas été tenue en compte. Comme l'idée est de proposer l'application dans les écoles de la communauté française, cette problématique est sans objet.

\subsubsection{Outils, concepts et environnement de travail}
\gls{Rsnap} est basé sur le langage \gls{snap} qui est un langage visuel. Ce choix s'explique par les objectifs qui sous-tendent ce travail, à savoir l'apprentissage de la programmation dans le but d'améliorer l'esprit logique des jeunes. L'accent étant mis sur la logique sous-jacente et, dans un premier temps, pas sur l'apprentissage d'un langage de programmation. Un but de l'application \gls{Rsnap} étant également de susciter des vocations, ces vocations se dirigeront d'elles-mêmes vers un vrai langage de programmation pour lequel il existe déjà beaucoup de cours.

Le fait de dissocier l'apprentissage de la logique et d'un langage de programmation était également important lors de ce travail. En effet, il faut diviser pour mieux régner et vouloir tout faire en même temps ne convient pas à tous. En se concentrant sur la logique, cela assure une plus grande confiance dans l'acquisition de la matière.

Un de nos objectifs était que les personnes qui guident l'activité n'aient pas besoin de connaissance particulière en programmation. De par cet objectif, un vrai langage de programmation a été écarté, car ils auraient dû au minimum avoir connaissance de sa syntaxe.\\

La manière dont les concepts sont abordés dans \gls{Rsnap} se fait par la procédure suivante :
\begin{itemize}
	\item une vidéo d'introduction à la mission ;
	\item un texte descriptif de la mission qui reprend les concepts théoriques qui seront introduits dans la mission ;
	\item une fois dans le programme les jeunes n'ont plus l'explication de manière explicite, cela les incite à travailler avec leur intuition tout en pouvant si nécessaire récupéré la description du point 2 ;
	\item une page d'aide est disponible pour chaque \gls{bloc} dans le menu contextuel.
\end{itemize}

En plus de cela, les missions implémentées dans ce travail sont de petites missions introduisant un à deux grands concepts maximum par mission. Ces petites missions ont pour but d'être assemblées pour faire un programme final plus grand. Plus d'information à propos du découpage et du contenu des missions est disponible dans la section \ref{missions}.\\

L'environnement de ce travail a été déterminer en grande partie par le milieu d'utilisation de l'application, à savoir les écoles. L'objectif d'indépendance des jeunes par rapport au référent a orienté l'activité vers la formation de binôme. Toute foi comme le groupe animé est une classe une introduction collective est possible et nécessaire. Ce travail étant basé sur une plate forme web, et l'accès au site n'étant pas limité, si les jeunes souhaitent utiliser le site en dehors des heures de cours, cela est également possible.
Nous avons donc un travail lors des séances collectives qui s'effectue par binôme avec une possibilité de travail en dehors des séances prévues à cet effet.

\subsubsection{Type d'organisation, enseignants, création des cours}
Le type d'organisation à l'heure actuelle est sans objet puisqu'aucune organisation officielle n'a été créée.\\

À propos des enseignants, les activités ont été dispensées par les auteurs de ce travail. Toute foi le projet a été créé dans le but que la personne qui dispense l'activité n'ait pas besoin de connaissance spécifique en programmation. Le simple fait de réaliser les projets avant les jeunes devrait être suffisant pour acquérir la logique nécessaire pour la transmettre.\\

La création des cours est un point sur lequel \gls{Rsnap} se distingue de beaucoup d'autres par le fait qu'une partie des missions existent déjà et fond partie de ce travail. Mais les professeurs peuvent également créer des missions et les partager avec le reste de la communauté. Tout comme ils peuvent également reprendre des missions existantes et les améliorer ou les adapter. %TODO voir s’il faut parler du fait qu'on chapeautera le tout.
