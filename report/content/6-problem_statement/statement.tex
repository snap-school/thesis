\chapter{Définition de la problématique}
Ce chapitre a pour but de situer Rsnap par rapport aux initiatives similaires. Premièrement, Rsnap est positionné en reprenant les concepts différenciateurs du chapitre précédent \ref{concept-differenciateur}. Ensuite, les besoins que la plateforme doit remplir sont abordés. Le troisième chapitre traite des choix technologiques qui ont été pris dans le cadre de ce travail.

\section{Positionnement de Rsnap}
%XXX essais 1 (simon) pas beaucoup mieux que le précédent mais différent
Rsnap est la plateforme créée pour ce travail. Elle est conçue pour être l'outil privilégié pour faire interagir les professeurs avec leur élèves dans l'apprentissage de la programmation.
Cette première partie du chapitre présente qui, comment et grâce à qui est utilisé Rsnap.

%TODO plateforme mal défini car web cours pour les enseignants
%TODO penser a une subsection
%TODO Intro trop complette ne reprendre que les sous titres
% Rsnap veut combler des lacunes mises en évidence dans l'approche internationale \ref{monde}. En effet, aucune plateforme  web n'est disponible en français et peu se veulent orientées vers les professeurs. Sur le plan du langage, la plateforme Rsnap se veut entièrement en français pour pouvoir être utilisée par tous les enfants qui savent lire et étant inscrits dans l'enseignement francophone. Rsnap se veut aussi tourné vers les professeurs. Cela se traduit par une aide à la gestion de classe, une indépendance des enfants par rapport à un référent, l'absence de prérequis pour le professeur et l'édition de missions communautaires.\\
% 
% La suite de ce chapitre reprend les concepts abordés dans le chapitre \ref{concepts} et y positionne Rsnap.

\subsection{Âge, genre et origine des utilisateurs}
Le positionnement RSnap vis à vis de son public visé est présenté dans cette section.

\paragraph{Âge}
Rsnap se veut une application qui vise un public de 10 à 14 ans. Les missions qui ont été développées dans le cadre de ce travail ciblent particulièrement cette tranche d'âge. Cette question est nuancée par l'analyse des résultats des expérimentations abordée dans le chapitre \ref{trancheage}. %TODO faire la référence quand on écrira les tranches d'âge
Toute fois, l'application permettant la création de mission de manière autonome, d'autres tranches d'âge peuvent s'y appliquer. La créativité des utilisateurs est la seule limite formelle.

\paragraph{Origine et genre}
En ce qui concerne les particularités des utilisateurs, aucune discrimination dans quelque sens que ce soit n'a été faite lors de la conception de ce travail, tant pour attirer un public que pour l'exclure. 

Comme l'idée est de proposer l'application dans les écoles de la Fédération Wallonie-Bruxelles, elle a été pensée pour un public francophone. 

\subsection{Outils, concepts et environnement de travail}
Cette section explique comment sont utilisés les différentes ressources pour apprendre la programmation aux enfants.

\paragraph{Outils}
\label{outil}
L'apprentissage de la programmation dans le but d'améliorer l'esprit logique des jeunes est un des objectifs principaux de ce travail. Dans cette optique, Rsnap se base sur un langage visuel \ref{languages}. L'accent est mis sur l'acquisition d'un esprit logique grâce à des exercices de logiques plutôt que sur l'apprentissage d'un langage de programmation.  En effet, suivant l'adage "il faut diviser pour mieux régner", vouloir tout faire en même temps ne convient pas à tous. En se concentrant sur la logique, cela assure une meilleure acquisition de la matière. De plus, un des buts de l'application Rsnap est aussi de susciter des vocations en programmation, ce qui se fera naturellement.

Un des objectifs de ce travail vise à ce que les personnes qui guident l'activité n'aient pas besoin de connaissance particulière en programmation. Un vrai langage de programmation a été écarté, car les encadrants auraient dû au minimum avoir connaissance de sa syntaxe.

\paragraph{Concepts}
La manière dont les concepts sont abordés dans Rsnap se fait par la procédure suivante :
\begin{itemize}
	\item une vidéo d'introduction à la mission ;
	\item un texte descriptif de la mission qui reprend les concepts théoriques introduits dans celle-ci ;
	\item une fois dans le programme, les jeunes n'ont plus d'explication explicite, ce qui les incite à travailler avec leur intuition tout en pouvant, si nécessaire, récupérer la description du point 2 ;
	\item une page d'aide est disponible pour chaque bloc dans le menu contextuel. %TODO bloc est il bien défini
\end{itemize}

Les missions implémentées dans ce travail sont de petites missions introduisant un à deux grands concepts maximum. Ces petites missions ont pour but d'être assemblées pour faire un programme final plus important. Plus d'informations à propos du découpage et du contenu des missions sont disponibles dans la section \ref{missions}.

\paragraph{environnement de ce travail}
L'environnement de ce travail est déterminé prioritairement par son milieu d'utilisation, à savoir les écoles. 

L'indépendance des jeunes par rapport au référent oriente l'activité vers la formation de binôme. Toute fois, comme le groupe animé est une classe, une introduction collective est possible et souhaitable. 
Si les jeunes le souhaite, ils peuvent également avoir accès au site web en dehors du cadre scolaire.

\subsection{Type d'organisation, enseignants, création des cours}
Cette dernière section explique comment se positionne Rsnap par rapport aux personnes qui créeront les missions.

\paragraph{Type d'organisation}
Il n'y a pas d'organisation officielle qui supporte Rsnap. Elle pourrait être soit reprise par le gouvernement, soit être indépendante mais travailler pour ce dernier.

\paragraph{Enseignants}
À propos des enseignants, le projet a été créé dans le but que la personne qui dispense l'activité n'ait pas besoin de connaissance spécifique en programmation. Le simple fait de réaliser les projets avant les jeunes devrait être suffisant pour acquérir la logique nécessaire à la transmettre.

\paragraph{Création des cours}
La création des cours est un point sur lequel Rsnap se distingue de beaucoup d'autres. En effet, une partie des missions existent déjà et est intégrée à ce travail. Les professeurs peuvent également créer des missions et les partager avec le reste de la communauté. Tout comme ils peuvent aussi reprendre des missions existantes et les améliorer ou les adapter. %TODO voir s’il faut parler du fait qu'on chapeautera le tout.

\section{Choix technologiques} %TODO revoir l'intro
Pour mettre en pratique les concepts du chapitre précédent, des critères d'utilisation ont du être déterminés. La première partie a pour but de les présenter.

Ensuite, deux principales fonctionnalités sont abordées. %TODO oui je sais c'est de nouveau trop court ;)
D'une part, une une plateforme pour aider les professeurs à gérer leurs classes. Et d'autre part, un langage de programmation utilisé pour les missions d'apprentissage. 
% La nécessité d'une plateforme légère et accessible partout, pour ne pas dépendre du matériel propre aux écoles, oriente le choix vers une plateforme web.
% Comme introduite dans le chapitre \ref{rails}, la technologie Rails convient. La suite de ce chapitre développe les besoins et choix techniques pris pour le développement de cette plateforme. 

% Comme expliqué au chapitre \ref{SNAP}, le choix du langage de programmation s'est porté sur SNAP! BYOB. Comme cette application existait déjà, il va donc falloir l'intégrer à la plateforme web. Ce sera le dernier point développé.



\subsection{Critères d'utilisation} %TODO pas sur que ce soit utile de rajouter ceci à la section précédente
Suite au positionnement de RSnap \ref{positionnement}, une analyse des besoins a été réalisée. Il en ressort une liste :

% La plateforme étant à destination des professeurs la majorité des besoins viennent d'eux. Cette partie va développer les besoins qui ont été pris en compte dans la conception de Rsnap. D'autres tierces %TODO ??? entrent également en jeu dans l'établissement du cahier des charges de la plateforme. Les enfants ou encore des constats tels que la faible qualité du matériel informatique dans les écoles apporte des contraintes supplémentaires. Les principaux besoins pris en compte sont:

%TODO subsection suivit direct par paragraph => pas bien
%TODO ordoner les besoins
% \paragraph{Disponibilité des exercices} fournir une série d'exercices aux élèves et permettre au professeur de les modifier ou d'en créer d'autres.
% \paragraph{Stocker les résolutions} permettre aux étudiants d'enregistrer leurs solutions et aux professeurs de récupérer ces travaux pour ensuite les corriger.

\begin{description}
  \item[Différencier les utilisateurs] afin de fournir une interface personnalisée suivant que l'utilisateur soit un professeur ou un élève.
  \item[Sauvegarder l'avancement] pour permettre à chaque utilisateur de connaitre où en est la résolution de ses missions.
  \item[Indépendance du matériel] afin d'avoir une plateforme accessible sur le plus d'ordinateurs et autres périphériques possibles.
  \item[Fiabilité et évolutivité] pour avoir la possibilité de rajouter des fonctionnalités tout en gardant une stabilité de l'application.
  \item[Facilité à mettre en place et à maintenir] en vue d'être déployable facilement dans une école si cette dernière a le matériel adéquat.
  \item[Accessibilité au plus grand nombre] et compréhensible pour le public visé.
  \item[Apprentissage de la programmation] pour fournir un environnement de développement intégré pour la pédagogie de la programmation.
\end{description}

\subsection{Plateforme d'apprentissage}
La plateforme d'apprentissage est la partie de l'application qui permet d'aider à gérer un ensemble d'étudiants en fournissant une interface pour mettre à disposition des missions et récupérer les soumissions.

Sur base des critères précités, les choix suivant ont été faits :
\begin{description}
  \item[Technologie web] Pour faciliter l'utilisation de la plateforme par tous, il est utile de choisir des technologies web. En effet, Il suffit d'un navigateur et d'une connexion internet pour que les utilisateurs puissent y accéder.
  \item[Logiciel libre] La connaissance ne devant pas être la propriété de quelques uns, il faut viser à la rendre accessible à tous. Les logiciels libres forcent à respecter ce droit d'accès à l'enseignement et à la culture. De plus, cela permet à tout un chacun d'héberger toute l'infrastructure en interne.
  \item[MVC] Pour séparer les différentes problématiques %TODO ??? changer problematique
  dans des composants bien définis, l'architecture MVC est un choix courant. Dans le cas de RSnap, le modèle permet de gérer les concepts d'utilisateurs, d'exercices et de résolutions. La possibilité de donner des droits différenciés aux étudiants ou aux professeurs se fait au niveau des contrôleurs. Les vues fournissent l'indépendance par rapport au matériel car elles peuvent être spécialisées en fonction de celui-ci.
\end{description}
Un logiciel qui permet de satisfaire tous ces choix en même temps est Rails. %TODO plus long
En effet, comme expliqué dans la section \ref{gems}, Rails peut bénéficier de différents gem's pour avoir facilement les fonctionnalités demandées. C'est aussi grâce à la connaissance approfondie de cet outil par les auteurs de ce travail que cette plateforme a été choisie. Ces deux éléments ont permis de développer rapidement et correctement une application qui correspond aux critères. 

%Rails permet de répondre ces besoins de manière élégante. En effet comme expliqué précédemment (\ref{rails}), rails permet de créer rapidement une application web.

%TODO subsection suivit direct par paragraph => pas bien 
% \paragraph{Web} Le fait que la solution soit une plateforme web permet de facilité son utilisation. Il suffit d'un navigateur, une connexion internet et que les utilisateurs créent leur compte. De plus, l'application étant développé en temps que logiciel libre, les écoles peuvent aussi héberger toute l'infrastructure en interne si elle le désire, moyennant un minimum de connaissance technique.
% \paragraph{MVC} Rails fourni une architecture MVC convenant très bien à la problématique. Le modèle permet de gérer les concepts d'utilisateur, d'exercices et de résolution. Ceci est encore facilité avec l'utilisation du gem \texttt{paperclip} pour les fichiers attachés.  La possibilité de donner différents droits aux utilisateurs suivant qu'ils sont des étudiants ou des professeurs se fait au niveau des contrôleurs. Ici encore la participation des gems : devise, rolify et authority facilite le travail. Les vues fournissent l'indépendance par rapport au matériel car elles ne nécessitent qu'un navigateur pour être affiché.
% \paragraph{Rails}La philosophie de Rails permet donc d'avoir une application qui se développe rapidement avec une architecture forte. Cette caractéristique permet de maintenir et de faire évoluer l'application facilement.



\subsection{Langage d'apprentissage}
Le langage d'apprentissage est utilisé dans l'application pour inculquer la programmation. En effet, c'est ce langage de programmation qu'utiliseront les enfants pour réaliser les missions.

Certains choix ont été faits pour tenir compte des critères d'utilisation :
\begin{description}
  \item[Langage complet] Pour pouvoir apprendre tout les concepts de programmation, il faut que le langage soit un langage complet dans le sens de Turing.  De plus, un langage impératif est intéressant car c'est le paradigme le plus répandu.
  \item[Blocs] Les enfants ont plus de facilité à s'approprier la matière quand elle se présente de manière visuelle. %TODO trouver référence pour visuelle
  Un langage par blocs est donc intéressant pour eux. Son interface colorée est attrayante. Son interface graphique fournit un retour direct des résultats de son programme à l'utilisateur. En effet, il peut voir évoluer son curseur dans la fenêtre dédiée.
  \item[JavaScript] Pour que le langage soit accessible au plus grand nombre, il faut qu'il soit disponible sur la majorité des navigateurs actuels des ordinateurs et des tablettes. Le standart du web actuel pour faire de l'animation est le JavaScript.
\end{description}
Ces différents choix mènent à sélectionner Snap! comme langage d'apprentissage pour ce travail.

%TODO subsection suivit direct par paragraph => pas bien
% \paragraph{web} Snap! est un langage basé sur JavaScript, n'importe quel navigateur relativement récent peut donc sans difficulté l'utiliser. Snap! est même concu pour supporter les tablets et donc les événements tactiles.
% \paragraph{langage complet} Pour pouvoir apprendre tout les concept de programmation, il faut que le langage soit un langage complet dans le sens de Turing. Un langage impératif est de plus intéressant car c'est le paradigme le plus répandu. Snap! permet d'enseigner ces concepts.
% \paragraph{amusant} Il faut absolument que les enfants s'amusent tout en apprenant. Snap! propose de programmer avec des blocs de couleurs ce qui rend attractif ce type de langage. De plus, le fait que directement, il est possible de faire bouger quelque chose à l'écran rajoute l'attrait que peuvent avoir les enfants.
