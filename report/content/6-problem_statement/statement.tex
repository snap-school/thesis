\chapter{Définition de la problématique}
Ce chapitre a pour but de situer \gls{rsnap} par rapport aux initiatives similaires. Premièrement, \gls{rsnap} est positionné en reprenant les concepts différenciateurs du chapitre précédent \ref{concepts}. Ensuite, les besoins que la plateforme doit remplir sont abordés. Le troisième chapitre traite des choix technologiques qui ont été pris dans le cadre de ce travail.

\section{Positionnement de \gls{rsnap}}
\label{positionnement}
\gls{rsnap} est la solution créée pour ce travail. Elle est conçue pour être l'outil privilégié pour faire interagir les professeurs avec leurs élèves dans l'apprentissage de la programmation.
Cette première partie du chapitre présente qui, comment et grâce à qui est utilisé \gls{rsnap}.

\subsection{Âge, genre et origine des utilisateurs}
Le positionnement \gls{rsnap} vis-à-vis de son public visé est présenté dans cette section.

\paragraph{Âge}
\gls{rsnap} se veut une application qui vise un public de 10 à 14 ans. Les \glspl{mission} qui ont été développées dans le cadre de ce travail ciblent particulièrement cette tranche d'âge. Cette question est nuancée par l'analyse des résultats des expérimentations abordée dans le chapitre \ref{trancheage}.
Toutefois, l'application permettant la création de \glspl{mission} de manière autonome, d'autres tranches d'âge peuvent s'y appliquer. La créativité des utilisateurs est la seule limite formelle.

\paragraph{Origine et genre}
En ce qui concerne les particularités des utilisateurs, aucune discrimination dans quelque sens que ce soit n'a été faite lors de la conception de ce travail, tant pour attirer un public que pour l'exclure.

Comme l'idée est de proposer l'application dans les écoles de la Fédération Wallonie-Bruxelles, elle a été pensée pour un public francophone.

\subsection{Outils, concepts et environnement de travail}
Cette section explique comment sont utilisées les différentes ressources pour apprendre la programmation aux enfants.

\paragraph{Outils}
\label{outil}
L'apprentissage de la programmation dans le but d'améliorer l'esprit logique des jeunes est un des objectifs principaux de ce travail. Dans cette optique, \gls{rsnap} se base sur un langage visuel \ref{langages}. L'accent est mis sur l'acquisition d'un esprit logique grâce à des exercices de logiques plutôt que sur l'apprentissage d'un langage de programmation.  En effet, suivant l'adage "il faut diviser pour mieux régner", vouloir tout faire en même temps ne convient pas à tous. En se concentrant sur la logique, cela assure une meilleure acquisition de la matière. De plus, un des buts de l'application \gls{rsnap} est aussi de susciter des vocations en programmation, ce qui se fera naturellement.

De plus, un des objectifs de ce travail vise à ce que les personnes qui guident l'activité n'aient pas besoin de connaissance particulière en programmation. Un vrai langage de programmation a été écarté, car les encadrants auraient dû au minimum avoir connaissance de sa syntaxe.

\paragraph{Concepts}
La manière dont les concepts sont abordés dans \gls{rsnap} se fait par la procédure suivante :
\begin{itemize}
	\item une vidéo d'introduction à la \gls{mission} ;
	\item un texte descriptif de la \gls{mission} qui reprend les concepts théoriques introduits dans celle-ci ;
	\item une fois dans le programme, les jeunes n'ont plus d'explication explicite, ce qui les incite à travailler avec leur intuition tout en pouvant, si nécessaire, récupérer la description du point 2 ;
	\item une page d'aide est disponible pour chaque \gls{bloc} dans le menu contextuel.
\end{itemize}

Les \glspl{mission} implémentées dans ce travail sont de petites \glspl{mission} introduisant un à deux grands concepts maximum. Ces petites \glspl{mission} ont pour but d'être assemblées pour faire un programme final plus important. Plus d'informations à propos du découpage et du contenu des \glspl{mission} sont disponibles dans la section \ref{missions}.

\paragraph{environnement de ce travail}
L'environnement de ce travail est déterminé prioritairement par son milieu d'utilisation, à savoir les écoles.

L'indépendance des jeunes par rapport au référent oriente l'activité vers la formation de binôme. Toutefois, comme le groupe animé est une classe, une introduction collective est possible et souhaitable.
Si les jeunes le souhaitent, ils peuvent également avoir accès au site web en dehors du cadre scolaire.

\subsection{Type d'organisation, enseignants, création des cours}
Cette dernière section explique comment se positionne \gls{rsnap} par rapport aux personnes qui créeront les \glspl{mission}.

\paragraph{Type d'organisation}
Il n'y a pas d'organisation officielle qui supporte \gls{rsnap}. Elle pourrait être soit reprise par le gouvernement, soit être indépendante, mais travailler pour ce dernier.

\paragraph{Enseignants}
À propos des enseignants, le projet a été créé dans le but que la personne qui dispense l'activité n'ait pas besoin de connaissance spécifique en programmation. Le simple fait de réaliser les projets avant les jeunes devrait être suffisant pour acquérir la logique nécessaire à la transmettre.

\paragraph{Création des cours}
La création des cours est un point sur lequel \gls{rsnap} se distingue de beaucoup d'autres. En effet, une partie des \glspl{mission} existent déjà et est intégrée à ce travail. Les professeurs peuvent également créer des \glspl{mission} et les partager avec le reste de la communauté. Tout comme ils peuvent aussi reprendre des \glspl{mission} existantes et les améliorer ou les adapter.

\section{Choix technologiques}
\label{techno}
Pour mettre en pratique les concepts du chapitre précédent, des critères d'utilisations ont dû être déterminés. La première partie a pour but de les présenter. Ensuite, les choix effectués à propos de la plateforme d'apprentissage sont développés suivis par ceux concernant le langage d'apprentissage.

\subsection{Critères d'utilisations}
Suite au positionnement de \gls{rsnap} \ref{positionnement}, un critère d'utilisations a été réalisé. Ces critères sont les suivants :
% La plateforme étant à destination des professeurs la majorité des besoins viennent d'eux. Cette partie va développer les besoins qui ont été pris en compte dans la conception de \gls{rsnap}. D'autres tiers %TODO ??? entrent également en jeu dans l'établissement du cahier des charges de la plateforme. Les enfants ou encore des constats tels que la faible qualité du matériel informatique dans les écoles apporte des contraintes supplémentaires. Les principaux besoins pris en compte sont:
% \paragraph{Disponibilité des exercices} fournir une série d'exercices aux élèves et permettre au professeur de les modifier ou d'en créer d'autres.
% \paragraph{Stocker les résolutions} permettre aux étudiants d'enregistrer leurs solutions et aux professeurs de récupérer ces travaux pour ensuite les corriger.

\begin{description}
  \item[Apprentissage de la programmation] pour fournir un environnement de développement intégré pour la pédagogie de la programmation.
  \item[Accessibilité au plus grand nombre] et compréhensible pour le public visé.
  \item[Différencier les utilisateurs] afin de fournir une interface personnalisée suivant que l'utilisateur soit un professeur ou un élève.
  \item[Indépendance du matériel] afin d'avoir une plateforme accessible sur le plus d'ordinateurs et autres périphériques possibles.
  \item[Fiabilité et évolutivité] pour avoir la possibilité de rajouter des fonctionnalités tout en gardant une stabilité de l'application.
  \item[Facilité à mettre en place et à maintenir] en vue d'être déployable facilement dans une école si cette dernière a le matériel adéquat à savoir un serveur web.
  \item[Sauvegarder l'avancement] pour permettre à chaque utilisateur de connaitre où en est la résolution de ses \glspl{mission}.
\end{description}

\subsection{Plateforme d'apprentissage}
La plateforme d'apprentissage est la partie de l'application qui permet d'aider à gérer un ensemble d'étudiants en fournissant une interface pour mettre à disposition des \glspl{mission} et récupérer les soumissions.

Sur base des critères précités, les choix suivants ont été faits :
\begin{description}
  \item[Technologie web] Pour faciliter l'utilisation de la plateforme par tous, il est utile de choisir des technologies web. En effet, il suffit d'un navigateur et d'une connexion internet pour que les utilisateurs puissent y accéder.
  \item[Logiciel libre] La connaissance ne devant pas être la propriété de quelques-uns, il faut viser à la rendre accessible à tous. Les logiciels libres forcent à respecter ce droit d'accès à l'enseignement et à la culture. De plus, cela permet à tout un chacun d'héberger l'infrastructure en interne.
  \item[MVC] Pour séparer les différentes problématiques dans des composants bien définis, l'architecture \gls{mvc} est un choix courant. Dans le cas de \gls{rsnap}, le modèle permet de gérer les concepts d'utilisateurs, d'exercices et de résolutions. La possibilité de donner des droits différenciés aux étudiants ou aux professeurs se fait au niveau des contrôleurs. Les vues fournissent l'indépendance par rapport au matériel, car elles peuvent être spécialisées en fonction de celui-ci.
\end{description}
Un logiciel qui permet de satisfaire tous ces choix en même temps est \gls{rails}. En effet, comme expliqué dans la section \ref{gems}, \gls{rails} peut bénéficier de différents \glspl{gem}  pour avoir facilement les fonctionnalités demandées. C'est aussi grâce à la connaissance approfondie de cet outil par les auteurs de ce travail que cette plateforme a été choisie. Ces deux éléments ont permis de développer rapidement et correctement une application qui correspond aux critères.

\subsection{Langage d'apprentissage}
Le langage d'apprentissage est utilisé dans l'application pour inculquer la programmation. En effet, c'est ce langage de programmation qu'utiliseront les enfants pour réaliser les \glspl{mission}.

Certains choix ont été faits pour tenir compte des critères d'utilisation :
\begin{description}
  \item[Langage complet] Pour pouvoir apprendre tous les concepts de programmation, il faut que le langage soit un langage complet dans le sens de Turing.  De plus, un langage impératif est intéressant, car c'est le paradigme le plus répandu.
  \item[Blocs] Les enfants ont plus de facilité à s'approprier la matière quand elle se présente de manière visuelle. %TODO trouver référence pour visuelle
  Un langage par \glspl{bloc} est donc intéressant pour eux. Son interface colorée est attrayante. Son interface graphique fournit un retour direct des résultats de son programme à l'utilisateur. En effet, il peut voir évoluer son curseur dans la fenêtre dédiée.
  \item[JavaScript] Pour que le langage soit accessible au plus grand nombre, il faut qu'il soit disponible sur la majorité des navigateurs actuels des ordinateurs et des tablettes. Le standard du web actuel pour faire de l'animation est le JavaScript.
\end{description}
Ces différents choix mènent à sélectionner \gls{snap} comme langage d'apprentissage pour ce travail.
