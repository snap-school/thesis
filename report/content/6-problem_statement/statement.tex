\chapter{Définition de la problématique}
\section{Positionnement de Rsnap}
Sur base des chapitres précédent, à savoir : les pratiques dans les autres pays développé dans le chapitre \ref{monde} et des concepts différenciateurs du chapitre \ref{concepts}. Cette partie est dédiée à explication précise de la problématique, aux apports de ce travail à l'état de l'art et au positionnement de Rsnap par rapport à ses homologues.

Comme développé dans le chapitre \ref{monde}, l'apprentissage de la programmation est déjà bien avancé dans plusieurs pays. Ce n'est malheureusement pas encore tout à fait le cas dans le notre. En effet quelques initiatives locales existe mais aucune décision politique n'a été prise jusqu'à présent.\\

A l'heure actuelle, aucune plateforme n'est disponible en français et de manière plus générale peu dans le monde se veulent orienté vers les professeurs. C'est pour combler ces lacunes que Rsnap a été pensé. Sur le plan du langage, la plate forme Rsnap se veut complètement en français pour pouvoir être utiliser par tous les enfants inscrit dans l'enseignement francophone sachant lire. Sur l'orientation professeur imprimée dans Rsnap, elle se traduit par une gestion de classe, une indépendance des enfants par rapport à un référent, l'absence de prérequis pour le professeur et l'édition de missions qui est communautaire.\\

La suite de ce chapitre va reprendre les concepts abordés dans le chapitre \ref{concepts} et détaillé ou Rsnap souhaite se positionner.

\subsection{Âge, origine et genre} 
Rsnap se veut une application qui vise un public de 10 à 14 ans et les missions qui ont été développées dans le cadre de ce travail, ciblent cette tranche d'âge particulièrement. Cette tranche sera nuancée lors de l'analyse des résultats de l'expérimentation sur les différentes tranches d'âge dans le chapitre \ref{trancheage}. %TODO faire la référence quand on écrira les tranches d'âge
Toute foi, l'application permettant la création de mission de manière autonome, il n'est pas exclu de créer des missions pour d'autres tranches d'âge. La créativité des utilisateurs est la seule limite formelle.\\

A propos des origines et des genres, aucune discrimination dans quelque sens que ce soit n'a été faite lors de la conception de ce travail. Tant pour attirer un public peu représenté que pour l'exclure. Comme l'idée est de proposer l'application dans les écoles de la communauté française, l'application a été pensée pour un public francophone. Hormis ce point linguistique c'est une problématique sans objet dans notre cas.

\subsection{Outils, concepts et environnement de travail} 
\label{SNAP}
Rsnap se base sur le langage Snap! BYOB qui est un langage visuel. Ce choix s'explique par les objectifs qui sous-tendent ce travail, à savoir l'apprentissage de la programmation dans le but d'améliorer l'esprit logique des jeunes. L'accent étant mis sur la logique sous-jacente et, dans un premier temps, pas sur l'apprentissage d'un langage de programmation. Un but de l'application Rsnap étant également de susciter des vocations, ces vocations se dirigeront d'elles-mêmes vers un vrai langage de programmation pour lequel il existe déjà beaucoup de cours.

Le fait de dissocier l'apprentissage de la logique et d'un langage de programmation était également important lors de ce travail. En effet, il faut diviser pour mieux régner et vouloir tout faire en même temps ne convient pas à tous. En se concentrant sur la logique, cela assure une plus grande confiance dans l'acquisition de la matière.

Un de nos objectifs était que les personnes qui guident l'activité n'aient pas besoin de connaissance particulière en programmation. De par cet objectif, un vrai langage de programmation a été écarté, car ils auraient dû au minimum avoir connaissance de sa syntaxe.\\

La manière dont les concepts sont abordés dans Rsnap se fait par la procédure suivante :
\begin{itemize}
	\item une vidéo d'introduction à la mission ;
	\item un texte descriptif de la mission qui reprend les concepts théoriques qui seront introduits dans la mission ;
	\item une fois dans le programme les jeunes n'ont plus l'explication de manière explicite, cela les incite à travailler avec leur intuition tout en pouvant si nécessaire récupéré la description du point 2 ;
	\item une page d'aide est disponible pour chaque bloc dans le menu contextuel.
\end{itemize}

En plus de cela, les missions implémentées dans ce travail sont de petites missions introduisant un à deux grands concepts maximum par mission. Ces petites missions ont pour but d'être assemblées pour faire un programme final plus grand. Plus d'information à propos du découpage et du contenu des missions est disponible dans la section \ref{missions}.\\

L'environnement de ce travail a été déterminer en grande partie par le milieu d'utilisation de l'application, à savoir les écoles. L'objectif d'indépendance des jeunes par rapport au référent a orienté l'activité vers la formation de binôme. Toute foi comme le groupe animé est une classe une introduction collective est possible et nécessaire. Ce travail étant basé sur une plate forme web, et l'accès au site n'étant pas limité, si les jeunes souhaitent utiliser le site en dehors des heures de cours, cela est également possible.
Nous avons donc un travail lors des séances collectives qui s'effectue par binôme avec une possibilité de travail en dehors des séances prévues à cet effet.

\subsection{Type d'organisation, enseignants, création des cours}
Le type d'organisation à l'heure actuelle est sans objet puisqu'aucune organisation officielle n'a été créée.\\

À propos des enseignants, les activités ont été dispensées par les auteurs de ce travail. Toute foi le projet a été créé dans le but que la personne qui dispense l'activité n'ait pas besoin de connaissance spécifique en programmation. Le simple fait de réaliser les projets avant les jeunes devrait être suffisant pour acquérir la logique nécessaire pour la transmettre.\\

La création des cours est un point sur lequel Rsnap se distingue de beaucoup d'autres par le fait qu'une partie des missions existent déjà et fond partie de ce travail. Mais les professeurs peuvent également créer des missions et les partager avec le reste de la communauté. Tout comme ils peuvent également reprendre des missions existantes et les améliorer ou les adapter. %TODO voir s’il faut parler du fait qu'on chapeautera le tout.

\section{Choix technologique}
Pour mettre en pratique les concepts du chapitre précédent et comme annoncé dans celui-ci, des choix technologique doivent être fait. Cette partie a pour but de les présenter. La nécessité d'une plateforme légère et accessible partout pour ne pas dépendre du matériel propre au école oriente le choix vers une plateforme web. 
Comme introduit dans le chapitre \ref{rails}, la technologie Rails parait convenir pour ceci, la suite de ce chapitre va développer les besoin et choix techniques pris dans le développement de cette platefome. 
Comme expliqué au chapitre \ref{SNAP}, le choix du langage de programmation s'est porté sur SNAP! BYOB. Comme cette application existait déjà, il va donc falloir l'intégré à la plateforme. Ce sera le point développé ensuite.


\paragraph{Platforme d'apprentissage}
La plateforme étant à destination des professeurs la majorité des besoins viennent d'eux. Cette partie va développé les besoins qui ont été pris en compte dans la conception de Rsnap. D'autres tierces entre également en jeu dans l'établissement du cahier des charge de la plateforme. Les enfants ou encore des constats tel que la faible qualité du matériel informatique dans les écoles apportent des contrainte supplémentaire. Les principales besoins pris en compte sont:
\begin{description}
  \item[disponibilité des exercices :] fournir une série d'exercices aux élèves et permettre au professeur de les modifier ou d'en créer d'autres ;
  \item[différencier les utilisateurs :] fournir une interface personnalisé suivant que l'utilisateur soit un professeur ou un élève ;
  \item[stocker les résolutions :] permettre aux étudiants d'enregistrer leurs solutions et aux professeurs de récupérer ces travaux pour ensuite les corriger ;
  \item[être indépendante du matériel :] avoir une platforme accessible sur le plus d'ordinateur possible ;
  \item[être fiable et évolutive :] avoir la possibilité de rajouter des fonctionnalités tout en gardant une stabilité de l'application ;
  \item[être facile à mettre en place et à maintenir :] pouvoir être déployer facilement dans une école si cette dernière est équiper pour.
\end{description}

Rails permet de répondre ces besoins de manière élégante. En effet comme expliqué précédemment (\ref{rails}), rails permet de créer rapidement une application web.

Rails fourni une architecture MVC convenant très bien à la problématique. Le modèle permet de gérer les concepts d'utilisateur, d'exercices et de résolution. Ceci est encore facilité avec l'utilisation du gem \texttt{paperclip} pour les fichiers attachés.  La possibilité de donner différents droits aux utilisateurs, suivant qu'ils sont des étudiants ou des professeurs se fait au niveau des contrôleurs. Ici encore la participation des gems : devise, rolify et authority facilite le travail. Les vues fournissent l'indépendance par rapport au matériel elles ne nécessite qu'un navigateur pour être afficher.\\

La philosophie de Rails permet donc d'avoir une application qui se développe rapidement avec une architecture forte. Cette caractéristique permet de maintenir et de faire évoluer l'application facilement.

Le fait que la solution soit une platforme web, permet de facilité son utilisation. Il suffit d'un navigateur, une connexion internet et que les utilisateurs créent leur compte. De plus, l'application étant développé en temps que logiciel libre, les écoles peuvent aussi héberger toute l'infrastructure en interne si elle le désire, moyennant les un minimum de connaissance technique.

\paragraph{Intégration du langage à la platforme}
