\chapter{Définition de la problématique}
Ce chapitre a pour but de situer Rsnap par rapport aux initiatives similaires. Premièrement, Rsnap sera positionné en reprenant les concepts différenciateurs du chapitre précédent. Ensuite, sont abordés les besoins que la plateforme web doit remplir. Le troisième chapitre traite des choix technologiques qui ont été pris dans le cadre de ce travail.

\section{Positionnement de Rsnap}
%TODO plateforme mal défini car web cours pour les enseignants
%TODO penser a une subsection
%TODO Intro trop complette ne reprendre que les sous titres
Rsnap veut combler des lacunes mises en évidence dans l'approche internationale \ref{monde}. En effet, aucune plateforme  web n'est disponible en français et peu se veulent orientées vers les professeurs. Sur le plan du langage, la plateforme Rsnap se veut entièrement en français pour pouvoir être utilisée par tous les enfants qui savent lire et étant inscrits dans l'enseignement francophone. Rsnap se veut aussi tourné vers les professeurs. Cela se traduit par une aide à la gestion de classe, une indépendance des enfants par rapport à un référent, l'absence de prérequis pour le professeur et l'édition de missions communautaires.\\

La suite de ce chapitre reprend les concepts abordés dans le chapitre \ref{concepts} et y positionne Rsnap.

\subsection{Âge, genre et origine des utilisateurs} 
Rsnap se veut une application qui vise un public de 10 à 14 ans. Les missions qui ont été développées dans le cadre de ce travail ciblent particulièrement cette tranche d'âge. Cette question est nuancée par l'analyse des résultats des expérimentations abordée dans le chapitre \ref{trancheage}. %TODO faire la référence quand on écrira les tranches d'âge
Toute fois, l'application permettant la création de mission de manière autonome, d'autres tranches d'âge peuvent s'y appliquer. La créativité des utilisateurs est la seule limite formelle.\\

En ce qui concerne les particularités des utilisateurs, aucune discrimination dans quelque sens que ce soit n'a été faite lors de la conception de ce travail, tant pour attirer un public que pour l'exclure. 

Comme l'idée est de proposer l'application dans les écoles de la Fédération Wallonie-Bruxelles, elle a été pensée pour un public francophone. 

\subsection{Outils, concepts et environnement de travail} 
\label{SNAP}
L'apprentissage de la programmation dans le but d'améliorer l'esprit logique des jeunes est un des objectifs principaux de ce travail. Dans cette optique, Rsnap se base sur l'\underline{outil} Snap! BYOB qui est un langage visuel \ref{languages}. L'accent est mis sur l'acquisition d'un esprit logique grâce à des exercices de logiques plutôt que sur l'apprentissage d'un langage de programmation.  En effet, suivant l'adage "il faut diviser pour mieux régner", vouloir tout faire en même temps ne convient pas à tous. En se concentrant sur la logique, cela assure une meilleure acquisition de la matière. De plus, un des buts de l'application Rsnap est aussi de susciter des vocations en programmation, ce qui se fera naturellement.

Un des objectifs de ce travail vise à ce que les personnes qui guident l'activité n'aient pas besoin de connaissance particulière en programmation. Un vrai langage de programmation a été écarté, car les encadrants auraient dû au minimum avoir connaissance de sa syntaxe.\\

La manière dont les \underline{concepts} sont abordés dans Rsnap se fait par la procédure suivante :
\begin{itemize}
	\item une vidéo d'introduction à la mission ;
	\item un texte descriptif de la mission qui reprend les concepts théoriques introduits dans celle-ci ;
	\item une fois dans le programme, les jeunes n'ont plus d'explication explicite, ce qui les incite à travailler avec leur intuition tout en pouvant, si nécessaire, récupérer la description du point 2 ;
	\item une page d'aide est disponible pour chaque bloc dans le menu contextuel. %TODO bloc est il bien défini
\end{itemize}

Les missions implémentées dans ce travail sont de petites missions introduisant un à deux grands concepts maximum. Ces petites missions ont pour but d'être assemblées pour faire un programme final plus important. Plus d'informations à propos du découpage et du contenu des missions sont disponibles dans la section \ref{missions}.\\

L'\underline{environnement de ce travail} est déterminé prioritairement par son milieu d'utilisation, à savoir les écoles. 

L'indépendance des jeunes par rapport au référent oriente l'activité vers la formation de binôme. Toute fois, comme le groupe animé est une classe, une introduction collective est possible et souhaitable. 
Si les jeunes le souhaite, ils peuvent également avoir accès au site web en dehors du cadre scolaire.

\subsection{Type d'organisation, enseignants, création des cours}
Il n'y a pas d'organisation officielle qui supporte Rsnap. Elle pourrait être soit reprise par le gouvernement, soit être indépendante mais travailler pour ce dernier.\\

À propos des enseignants, le projet a été créé dans le but que la personne qui dispense l'activité n'ait pas besoin de connaissance spécifique en programmation. Le simple fait de réaliser les projets avant les jeunes devrait être suffisant pour acquérir la logique nécessaire à la transmettre.\\

La création des cours est un point sur lequel Rsnap se distingue de beaucoup d'autres. En effet, une partie des missions existent déjà et est intégrée à ce travail. Les professeurs peuvent également créer des missions et les partager avec le reste de la communauté. Tout comme ils peuvent aussi reprendre des missions existantes et les améliorer ou les adapter. %TODO voir s’il faut parler du fait qu'on chapeautera le tout.

\section{Choix technologiques} %TODO revoir l'intro et la structure des titres.
Pour mettre en pratique les concepts du chapitre précédent, des choix technologiques ont du être faits. La première partie a pour but de les présenter. 

% La nécessité d'une plateforme légère et accessible partout, pour ne pas dépendre du matériel propre aux écoles, oriente le choix vers une plateforme web. %TODO deja trop expliqué
% Comme introduite dans le chapitre \ref{rails}, la technologie Rails convient. La suite de ce chapitre développe les besoins et choix techniques pris pour le développement de cette plateforme. 
La nécessité d'une plateforme légère et accessible partout oriente le choix vers une plateforme web. La suite de ce chapitre développe les choix techniques pris pour le développement de la plateforme Rails \ref{rails}. 

% Comme expliqué au chapitre \ref{SNAP}, le choix du langage de programmation s'est porté sur SNAP! BYOB. Comme cette application existait déjà, il va donc falloir l'intégrer à la plateforme web. Ce sera le dernier point développé.
Comme expliqué au chapitre \ref{SNAP}, le choix du langage de programmation s'est porté sur SNAP! BYOB.  Ce sera le dernier point développé.



\subsection{Analyse des besoins} %TODO pas sur que ce soit utile de rajouter ceci à la section précédente
Suite au positionnement de RSnap \ref{positionnement}, certains besoins techniques ont été mis en avant.

% La plateforme étant à destination des professeurs la majorité des besoins viennent d'eux. Cette partie va développer les besoins qui ont été pris en compte dans la conception de Rsnap. D'autres tierces %TODO ??? entrent également en jeu dans l'établissement du cahier des charges de la plateforme. Les enfants ou encore des constats tels que la faible qualité du matériel informatique dans les écoles apporte des contraintes supplémentaires. Les principaux besoins pris en compte sont:

%TODO subsection suivit direct par paragraph => pas bien
%TODO ordoner les besoins
% \paragraph{Disponibilité des exercices} fournir une série d'exercices aux élèves et permettre au professeur de les modifier ou d'en créer d'autres.
\paragraph{Différencier les utilisateurs} fournir une interface personnalisée suivant que l'utilisateur soit un professeur ou un élève.
% \paragraph{Stocker les résolutions} permettre aux étudiants d'enregistrer leurs solutions et aux professeurs de récupérer ces travaux pour ensuite les corriger.
\paragraph{Sauvegarder l'avancement} connaitre où en est la résolution des missions.
\paragraph{Indépendant du matériel} avoir une plateforme accessible sur le plus d'ordinateurs et autres périphériques possible possible.
\paragraph{Fiable et évolutif} avoir la possibilité de rajouter des fonctionnalités tout en gardant une stabilité de l'application.
\paragraph{Facile à mettre en place et à maintenir} pouvoir être déployé facilement dans une école si cette dernière est équiper pour.
\paragraph{Accessible au plus grand nombre} être compréhensible pour le public visé : enfants de 10 à 14 ans et professeur de tout âge.
\paragraph{Apprentissage de la programmation} fournir un environnement de développement intégré pour la pédagogie de la programmation.

\subsection{Plateforme d'apprentissage}
Pour résoudre les besoins précités, les choix suivant ont été fait.

\begin{description}
  \item[Technologie web] Pour faciliter l'utilisation de la platforme par tous, il est utile de choisir des technologies web. En effet, Il suffit d'un navigateur, une connexion internet pour que les utilisateurs puissent utiliser celle-ci.
  \item[Logiciel libre] La connaissance ne doit pas être la propriété de quelqu'un, il faut la rendre accessible à tous. Les logiciels libres forcent à respecter ce droit d'accès à l'enseignement et à la culture. De plus, cela permet à tout un chacun d'héberger toute l'infrastructure en interne si il le désire.
  \item[MVC] Pour séparer les différentes problématiques dans des composants bien défini, l'architecture MVC est un choix courant. Dans le cas de RSnap, Le modèle permet de gérer les concepts d'utilisateur, d'exercices et de résolution. La possibilité de donner différents droits aux utilisateurs suivant qu'ils sont des étudiants ou des professeurs se fait au niveau des contrôleurs. Les vues fournissent l'indépendance par rapport au matériel car elles peuvent être spécialisée en fonction de celui-ci.
\end{description}
Rails permet de résoudre ces différents choix. %TODO plus long
En effet, comme expliqué dans la section \ref{gems}, Rails peut bénéficier de différents gem's pour avoir facilement les fonctionnalités demandées. Mais c'est aussi car les auteurs de ce travail avait l'habitude de travailler avec Rails que cette platforme fut choisie. Cette ensemble permet de développer rapidement et correctement une application qui correspond aux besoins. 

%Rails permet de répondre ces besoins de manière élégante. En effet comme expliqué précédemment (\ref{rails}), rails permet de créer rapidement une application web.

%TODO subsection suivit direct par paragraph => pas bien 
% \paragraph{Web} Le fait que la solution soit une plateforme web permet de facilité son utilisation. Il suffit d'un navigateur, une connexion internet et que les utilisateurs créent leur compte. De plus, l'application étant développé en temps que logiciel libre, les écoles peuvent aussi héberger toute l'infrastructure en interne si elle le désire, moyennant un minimum de connaissance technique.
% \paragraph{MVC} Rails fourni une architecture MVC convenant très bien à la problématique. Le modèle permet de gérer les concepts d'utilisateur, d'exercices et de résolution. Ceci est encore facilité avec l'utilisation du gem \texttt{paperclip} pour les fichiers attachés.  La possibilité de donner différents droits aux utilisateurs suivant qu'ils sont des étudiants ou des professeurs se fait au niveau des contrôleurs. Ici encore la participation des gems : devise, rolify et authority facilite le travail. Les vues fournissent l'indépendance par rapport au matériel car elles ne nécessitent qu'un navigateur pour être affiché.
% \paragraph{Rails}La philosophie de Rails permet donc d'avoir une application qui se développe rapidement avec une architecture forte. Cette caractéristique permet de maintenir et de faire évoluer l'application facilement.



\subsection{Langage d'apprentissage}
%TODO intro

%TODO subsection suivit direct par paragraph => pas bien
\paragraph{web} Snap! est un langage basé sur JavaScript, n'importe quel navigateur relativement récent peut donc sans difficulté l'utiliser. Snap! est même concu pour supporter les tablets et donc les événements tactiles.
\paragraph{langage complet} Pour pouvoir apprendre tout les concept de programmation, il faut que le langage soit un langage complet dans le sens de Turing. Un langage impératif est de plus intéressant car c'est le paradigme le plus répandu. Snap! permet d'enseigner ces concepts.
\paragraph{amusant} Il faut absolument que les enfants s'amusent tout en apprenant. Snap! propose de programmer avec des blocs de couleurs ce qui rend attractif ce type de langage. De plus, le fait que directement, il est possible de faire bouger quelque chose à l'écran rajoute l'attrait que peuvent avoir les enfants.
