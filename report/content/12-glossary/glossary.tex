% http://en.wikibooks.org/wiki/LaTeX/Glossary

% \newglossaryentry{}
% {
%  name={},
%  description={}
% }

\newglossaryentry{rsnap}
{
 name=Rsnap,
 description={Rsnap \cite{rsnap} est l'application développée dans le cadre de ce travail. Elle est visible sur \url{https://rsnap.herokuapp.com/}. Le nom Rsnap provient de la contraction des deux principaux outils utilisé pour construire cette plateforme à savoir : \gls{rails} et \gls{snap}.}
}
\newglossaryentry{pull}
{
 name={pull request},
 description={Une \texttt{pull request} est une méthode de soumission de contribution à un projet au développement ouvert. C'est  souvent la meilleure façon de soumettre des contributions à un projet à l'aide d'un système de contrôle de version distribué , comme git.}
}
\newglossaryentry{rails}
{
 name={Ruby on Rails},
 text={Rails},
 first={Ruby on Rails (Rails)},
 description={Ruby on Rails \cite{rails} est un framework web.}
}
\newglossaryentry{snap}
{
 name={Snap! Build Your Own Blocks},
 text={Snap!},
 first={Snap! BYOB (Snap!)},
 description={Snap! \cite{snap} est un langage de programmation visuel par glisser-déposer de \glspl{bloc}.}
}
\newglossaryentry{gem}
{
 name={gem},
 description={Un gem \cite{gem} est un programme ou une bibliothèque Ruby distribuée grâce à RubyGems.}
}
\newglossaryentry{mvc}
{
 name={modèle-vue-contrôleur},
 text={MVC},
 first={modèle-vue-contrôleur (MVC)},
 description={Le patron modèle-vue-contrôleur \cite{wiki-mvc} est un patron destiné à répondre aux besoins des applications interactives en séparant les problématiques liées aux différents composants au sein de leur architecture respective.}
}
\newglossaryentry{bloc}
{
 name={bloc},
 description={Un bloc est une pièce de puzzle symbolisant une opération qui est utilisée pour créer un \gls{script}.}
}
\newglossaryentry{mission}
{
 name={mission},
 description={Une mission est un exercice mettant en scène un problème.}
}
\newglossaryentry{script}
{
 name={script},
 description={Un script est un ensemble de blocs créant un programme exécutable.}
}
\newglossaryentry{sprite}
{
 name={sprite},
 description={Un sprit est une entité graphique contenant un ou plusieurs scripts associés.}
}
\newglossaryentry{rest}
{
 name={representational state transfer},
 text={REST},
 first={representational state transfer (REST)},
 description={REST est un style d’architecture pour les systèmes hypermédia distribués. Il impose une communication sans états et une interface uniforme.}
}
\newglossaryentry{role}
{
 name={rôle},
 description={Type de personne utilisant \gls{rsnap}. Actuellement, deux rôles sont définis : élève et professeur.}
}
\newglossaryentry{primaire} %TODO
{
 name={primaire},
 description={L'enseignement primaire est la première partie de l'enseignement officiel obligatoire pour les enfants typiquement de 6 à 12 ans en Belgique.}
}
\newglossaryentry{secondaire} %TODO
{
 name={secondaire},
 description={L'enseignement primaire est la seconde partie de l'enseignement officiel obligatoire pour les enfants typiquement de 12 à 18 ans en Belgique.}
}
\newglossaryentry{humanite} %TODO
{
 name={humanité},
 see=secondaire,
 description={}
}
\newglossaryentry{fondamental} %TODO
{
 name={fondamental},
 see=primaire,
 plural={fondamentales},
 description={}
}
