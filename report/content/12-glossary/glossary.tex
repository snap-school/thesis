% http://en.wikibooks.org/wiki/LaTeX/Glossary

% \newglossaryentry{}
% {
%  name={},
%  description={}
% }

\newglossaryentry{Rsnap}
{
 name=Rsnap,
 description={Rsnap \cite{rsnap} est l'application développée dans le cadre de ce travail. Elle est visible sur \url{https://rsnap.herokuapp.com/}}
}
\newglossaryentry{pull}
{
 name={pull request},
 description={Une \texttt{pull request} est une méthode de soumission de contribution à un projet au développement ouvert. C'est  souvent la meilleure façon de soumettre des contributions à un projet à l'aide d'un système de contrôle de version distribué , comme Git}
}
\newglossaryentry{rails}
{
 name={Ruby on Rails},
 text={Rails},
 first={Ruby on Rails (Rails)},
 description={Ruby on Rails \cite{rails} est un framework web}
}
\newglossaryentry{snap}
{
 name={Snap! Build Your Own Blocks},
 text={Snap!},
 first={Snap! BYOB (Snap!)},
 description={Snap! \cite{snap} est un langage de programmation visuel par glissé-déposé de \glspl{bloc}}
}
\newglossaryentry{gem}
{
 name={gem},
 description={Un gem \cite{gem} est un programme ou une bibliothèque Ruby distribuée gràce à RubyGems}
}
\newglossaryentry{fondamental} %TODO en france on dit primaire
{
 name={fondamental},
 description={}
}
\newglossaryentry{secondaire}
{
 name={secondaire},
 description={}
}
\newglossaryentry{mvc}
{
 name={modèle-vue-controleur},
 text={MVC},
 first={modèle-vue-controleur (MVC)},
 description={Le patron modèle-vue-contrôleur \cite{wiki-mvc} est un patron destiné à répondre aux besoins des applications interactives en séparant les problématiques liées aux différents composants au sein de leur architecture respective}
}
\newglossaryentry{bloc}
{
 name={bloc},
 description={Un bloc est une forme de pièce de puzzle qui est utilisée pour créer un programme} %TODO en anglais ca donne mieux http://wiki.scratch.mit.edu/wiki/Blocks
}
\newglossaryentry{mission}
{
 name={mission},
 description={}
}
